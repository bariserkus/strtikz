\pgfkeys{
 /lumpedmass/.is family, /lumpedmass,
  default/.style = {number of storys = 5,
	story height = 1cm,
	startX = 0cm,
	startY = 0cm,
	frame line thickness = 1pt,
	support width = 0.3cm,
	support height = 0.15cm,
	support line thickness = 1.0pt,
	show supports = 1,
	isolator width = 0.3cm,
	isolator thickness = 0.2cm,
	isolator line thickness = 1.5pt,
	foundation thickness = 0.5cm,
	mass radius = 2pt,
	show mass = 1,
	show axes = 1,
	base slab width=0.8cm,
	base slab height=0.3cm,
	foundation width=1cm,
	x axis length=0.5cm,
	y axis length=0.5cm,
	show deflected shape = 1,
	drift distance=0.4cm,
	drift curve ratio x=0.65,
	drift curve ratio y=0.65,
	base slab drift=1.1cm,
	foundation drift=2cm,
	rotation dof radius = 0.4cm,
	rotation dof text position = 0.8,
	arrow tip length=3pt,
	arrow tip width=2pt,
	story to place superstructure dof=2,
	text for the superstructure dof=$x_i(t)$,
	text for the superstructure rotation dof=$x_i(t)$,
	text for the isolation dof=$x_b(t)$,
	text for the ground dof=$x_g(t)$,
	text for the story mass=$m_i$,
	text for the base slab mass=$m_b$,
	text for the story properties={$k_i,c_i$},
	text location for mass and story properties=left,
	story dof arrow tick = 6pt,
	vertical dof ref line start ratio=0.6,
	show superstructure dofs = 1,
	show superstructure rotation dofs = 1,
	show ground drift = 1,},
	number of storys/.store in = \nstory,
	story height/.store in = \storyheight,
	startX/.store in = \startx,
	startY/.store in = \starty,
	frame line thickness/.store in = \framelinet,
	support width/.store in = \supw,
	support height/.store in = \suph,
	support line thickness/.store in = \baselinet,
	show supports/.store in = \showss,
	isolator width/.store in = \isow,
	isolator thickness/.store in = \isot,
	isolator line thickness/.store in = \isolinet,
	foundation thickness/.store in = \foundt,
	mass radius/.store in = \massrad,
	show mass/.store in = \showmass,
	show axes/.store in = \showaxes,
	base slab width/.store in=\baseslabw,
	base slab height/.store in=\baseslabh,
	foundation width/.store in=\foundationw,
	x axis length/.store in=\axeslenX,
	y axis length/.store in=\axeslenY,
	show deflected shape/.store in=\showdeflectedshape,
	drift distance/.store in=\driftdist,
	drift curve ratio x/.store in=\driftcurveratiox,
	drift curve ratio y/.store in=\driftcurveratioy,
	base slab drift/.store in=\basedriftdist,
	foundation drift/.store in=\foundationdrift,
	rotation dof radius/.store in=\rotdofradius,
	rotation dof text position/.store in=\rotdoftextpos,
	arrow tip length/.store in=\arrlen,
	arrow tip width/.store in=\arrwid,
	story to place superstructure dof/.store in=\doftextstory,
	text for the superstructure dof/.store in=\doftext,
	text for the superstructure rotation dof/.store in=\rotationdoftext,
	text for the isolation dof/.store in=\isodoftext,
	text for the ground dof/.store in=\grounddoftext,
	text for the story mass/.store in=\storymasstext,
	text for the base slab mass/.store in=\basemasstext,
	text for the story properties/.store in=\storyproptext,
	text location for mass and story properties/.store in=\massproploc,
	story dof arrow tick/.store in=\storydofendtick,
	vertical dof ref line start ratio/.store in=\vertreflineratio,
	show superstructure dofs/.store in=\showsuperstructuredof,
	show superstructure rotation dofs/.store in=\showsuperstructurerotationdof,
	show ground drift/.store in=\showgrounddrift,}
\newcommand{\lumpedmass}[1][]{
\pgfkeys{/lumpedmass, default, #1}
\tikzmath{
int \nstory, \ncol, \nlev, \nlevmo, \ncolmo, \iii, \jjj, \kkk, \showss;
real \storyheight, \startx, \starty, \xx, \y;
real \supw, \suph, \isow, \isot, \foundt, \massrad;
real \axissp, \framelinet, \baselinet, \isolinet;
real \rigbasestartx, \rigbaseendx, \isoboty, \isotopy;
real \foundboty, \foundtopy, \foundstartx, \foundendx;
int \showmass, \showdeflectedshape;
real \arrowlenratio, \minlen, \dofxx, \dofyy, \arrlen, \arrrad;
real \driftdist, \driftcurve, \driftcurveratioy, \driftcurveratiox, \rotdofradius;
real \arrlen, \arrwid;
int \showsuperstructuredof, \doftextstory, \doftextfloor;
real \grdofarrstartx, \grdofarrstarty, \grdofarrendx, \grdofarrendy, \grdoftickboty, \grdofticktopy;
real \vertrefineratio;
int \showgrounddrift;
real \tempmasslimit;
int \starttickstatus;
%Conversion all units to the unit pt
\storyheight = \storyheight;
\startx = \startx;
\starty = \starty;
\supw = \supw;
\suph = \suph;
\isow = \isow;
\isot = \isot;
\foundt = \foundt;
\framelinet = \framelinet;
\baselinet = \baselinet;
\isolinet = \isolinet;
\massrad = \massrad;
\baseslabw = \baseslabw;
\baseslabh = \baseslabh;
\foundationw = \foundationw;
\axeslenX = \axeslenX;
\axeslenY = \axeslenY;
\driftdist = \driftdist;
\driftcurveratiox = \driftcurveratiox;
\driftcurveratioy = \driftcurveratioy;
\basedriftdist = \basedriftdist;
\foundationdrift = \foundationdrift;
\arrlen = \arrlen;
\arrwid = \arrwid;
\storydofendtick = \storydofendtick;
\rotdofradius =\rotdofradius;
%End conversion, All values are now in pt units.
\axissp = 0.2cm;
\baywidth = 0;  %Set for a lumped mass system
\nbays = 0;  %Set for a lumped mass system
\ncol = \nbays+1; %number of columns
\nlev = \nstory+1; %number of levels
if \nstory>1 then {\nlevmo = \nlev-1;} else {\nlevmo=2;};
if \nbays>1 then {\ncolmo = \ncol-1;} else {\ncolmo=2;};
for \iii in {1,...,{\nlev}}{
\y{\iii} = (\iii-1)*\storyheight;
for \jjj in {1,...,{\ncol}}{
\x{\jjj} = (\jjj-1)*\baywidth;
};
};
\rigbasestartx = \x1-\baseslabw/2;
\rigbaseendx = \x{\ncol}+\baseslabw/2;
\isoboty = -\baseslabh-\baselinet/2-\isot;
\isotopy = -\baseslabh-\baselinet/2;
\foundboty = -\baseslabh-\baselinet-\isot-\foundt;
\foundtopy = -\baseslabh-\baselinet-\isot;
\foundstartx = \x1-\foundationw/2;
\foundendx = \x{\ncol}+\foundationw/2;
if \showss==0 then {\grdofarrstartx = 0;
\grdofarrstarty = -\suph/2;
\grdofarrendx = \foundationdrift;
\grdofarrendy = -\suph/2;
\grdoftickboty = -\suph*0.75;
\grdofticktopy = -\suph*0.25;} else {};
if \showss==1 then {\grdofarrstartx = 0;
\grdofarrstarty = -\suph/2;
\grdofarrendx = \foundationdrift;
\grdofarrendy = -\suph/2;
\grdoftickboty = -\suph*0.75;
\grdofticktopy = -\suph*0.25;} else {};
if \showss==2 || \showss==3   then {\grdofarrstartx = 0;
\grdofarrstarty = -\baseslabh/2;
\grdofarrendx = \foundationdrift;
\grdofarrendy = -\baseslabh/2;
\grdoftickboty = -\baseslabh*0.75;
\grdofticktopy = -\baseslabh*0.25;} else {};
if \showss==4 then {\grdofarrstartx = 0;
\grdofarrstarty = \foundboty+\foundt*0.5;
\grdofarrendx = \foundationdrift;
\grdofarrendy = \foundboty+\foundt*0.5;
\grdoftickboty = \foundboty+\foundt*0.25;
\grdofticktopy = \foundboty+\foundt*0.25 + \foundt*0.5;}
else{\basedriftdist=0;};
%%%%%Axes%%%%%
\Xaxesstarty = \y{1};
\Yaxesstartx = \x{1};
\Yaxesstarty = \y{\nlev}+\axissp;
if equal(\showss,0) then {\Xaxesstartx = \x{\ncol}+\axissp;}         else {\Xaxesstartx = \x{\ncol}+\axissp+\supw/2;};
if equal(\showss,1) then {\Xaxesstartx = \x{\ncol}+\axissp+\supw/2;} else {\Xaxesstartx = \x{\ncol}+\axissp+\supw;};
%%%%%%%Deflected Shape%%%%%%%%%%%%%%%%%
\driftdist = \driftdist;
\driftcurvy = \driftcurveratioy*\storyheight;
\driftcurvx = \driftcurveratiox*\driftdist;
%%%%%%%DOFs%%%%%%%%%%
\doftextfloor = \doftextstory+1;
\vertrefstart = \vertreflineratio*\driftcurvy;
\storydofshift = \vertreflineratio*\driftcurvy;
%%%%%%%Check if foundation drift is very small such that start tick clashed with the mass%%%%%%%
\tempmasslimit = 1.2*\massrad;
if (\foundationdrift < \tempmasslimit) then {\starttickstatus=1;} else {\starttickstatus=0;};
}%End tikzmath

\begin{scope}[x=1pt, y=1pt, xshift=\startx, yshift=\starty, rotate=0]; % Drawing everything in pt units

\tikzset{arrowbe/.style = {arrows={->[length=\arrlen, width=\arrwid, flex=1, bend]}}};
\tikzset{arrowben/.style = {arrows={<->[length=\arrlen, width=\arrwid, flex=1, bend]}}};


%%%%%% Undeflected Shape%%%%%%%%%%%%%%%%%%%%%%%%%%%%%%%%%%%%%%%%%%%%%%
% Draw the line
\foreach \iii in {2,...,{\nlev}}{
\draw [line width = \framelinet] (\x{1},\usevar\y{\iii-1}) -- (\x{1},\usevar\y{\iii});
}
% Show Mass
\ifthenelse{\showmass=1}{
\foreach \iii in {2,...,{\nlev}}
{\foreach \jjj in {1,...,{\ncol}}{
\shade[ball color=black] (\x{\jjj},\y{\iii}) circle (\massrad) ;
}}
}{}

% Add text for mass and stiffness and damping
\path (\x{1},\y{\doftextfloor})node[inner sep=2pt, outer sep=\massrad, \massproploc, opaque]{\storymasstext} 
++(0,-\storyheight/2)node[inner sep=2pt, \massproploc, opaque]{\storyproptext};


%%% Different types of supports
\ifthenelse{\showss=0}{
}{}

\ifthenelse{\showss=1}{
\foreach \jjj in {1,...,{\ncol}}
{
\fill [gray] (\x{\jjj},\y{1}) +({-\supw/2},-\suph) rectangle +({\supw/2},\y{1});
\draw [line width = \baselinet] (\x{\jjj},\y{1}) +({-\supw/2},\y{1}) -- +({\supw/2},\y{1});
}
}{}

\ifthenelse{\showss=2}{
\foreach \jjj in {1,...,{\ncol}}
{
\fill [gray] (\rigbasestartx, 0) rectangle (\rigbaseendx,-\baseslabh);
\draw [line width = \baselinet] (\rigbasestartx, 0) -- (\rigbaseendx,0);
}
}{}

\ifthenelse{\showss=3}{
\foreach \jjj in {1,...,{\ncol}}
{\fill [fill=gray, draw=black, line width = \baselinet] (\rigbasestartx, 0) rectangle (\rigbaseendx,-\baseslabh);}
}{}

\ifthenelse{\showss=4}{
\path(\baseslabw/2,0) node[above=-2pt]{\basemasstext};

\fill [fill=gray, draw=black, line width = \baselinet] (\rigbasestartx, 0) rectangle (\rigbaseendx,-\baseslabh);

\foreach \jjj in {1,...,{\ncol}}
{\filldraw [fill = white!20!black, draw=black, line width=\isolinet]
(\x{\jjj},\y{1}) +(-\isow/2,\isoboty) rectangle +(\isow/2,\isotopy);}

\fill [fill=gray]
(\foundstartx, \foundboty) rectangle (\foundendx,\foundtopy);
\draw [line width = \baselinet] (\foundstartx, \foundtopy) -- (\foundendx, \foundtopy);
}{}
%%%%%%%%%%%%%%%%%%%%%%%%%%%%%%%%%%%%%%%%%%%%%% 

%%%%%%%%%%% Deflected Shape %%%%%%%%%%%%%%%%%%%%%%%%%%%%%%%%%
\ifthenelse{\showdeflectedshape=1}{
\begin{scope}[xshift = \foundationdrift]

\begin{scope}[xshift = \basedriftdist]

%Deflected Shape
\foreach \iii in {2,...,{\nlev}}{
\draw [line width = \framelinet, gray] (\x{1}+\driftdist*\useeval{\iii-2},\usevar\y{\iii-1})
..controls++(\driftcurvx,\driftcurvy) and ++(-\driftcurvx,-\driftcurvy)..(\x{1}+\driftdist*\useeval{\iii-1},\usevar\y{\iii});
}

% Show Mass
\ifthenelse{\showmass=1}{
\foreach \iii in {2,...,{\nlev}}
{\foreach \jjj in {1,...,{\ncol}}{
\shade[ball color = gray] (\x{\jjj}+\driftdist*\useeval{\iii-1},\y{\iii}) circle (\massrad);
}}
}{}


%%%%%%%%%%%%%%%%%%%%%Support Conditions

%%%% Support Type = 0

\ifthenelse{\showss=0}{
}{}

%%%% Support Type = 1

\ifthenelse{\showss=1}{
\foreach \jjj in {1,...,{\ncol}}
{
\fill [gray] (\x{\jjj},\y{1}) +({-\supw/2},-\suph) rectangle +({\supw/2},\y{1});
\draw [line width = \baselinet] (\x{\jjj},\y{1}) +({-\supw/2},\y{1}) -- +({\supw/2},\y{1});
}
}{}

%%%% Support Type = 2

\ifthenelse{\showss=2}{
\foreach \jjj in {1,...,{\ncol}}
{
\fill [gray] (\rigbasestartx, 0) rectangle (\rigbaseendx,-\baseslabh);
\draw [line width = \baselinet] (\rigbasestartx, 0) -- (\rigbaseendx,0);
}
}{}

%%%% Support Type = 3

\ifthenelse{\showss=3}{
\foreach \jjj in {1,...,{\ncol}}
{\fill [fill=gray, draw=black, line width = \baselinet]
(\rigbasestartx, 0) rectangle (\rigbaseendx,-\baseslabh);}
}{}

%%%% Support Type = 4

\ifthenelse{\showss=4}{

\fill [fill=gray, draw=gray, line width = \baselinet, fill opacity = 0.35]
(\rigbasestartx, 0) rectangle (\rigbaseendx,-\baseslabh);

\foreach \jjj in {1,...,{\ncol}}
{\filldraw [fill = white!75!black, draw=gray, line width=\isolinet, opacity = 1]
(\x{\jjj},\y{1}) ++(\isow/2,\isotopy) -- ++(-\basedriftdist,-\isot) -- ++(-\isow,0) -- ++(\basedriftdist, \isot) -- cycle;}

%Isolaton DOF
\draw [arrowbe] (-\basedriftdist,-\baseslabh/2) --
node[above=-2pt, xshift =-\baseslabw/4, midway]{\isodoftext}(0,-\baseslabh/2);
}{}



%%%%%%%%%%%%%%%%% Super Structure DOF Arrows

%%%%% SuperStructure DOF = 1

\ifthenelse{\showsuperstructuredof=1}{
\foreach \iii in {2,...,{\nlev}}{


% Place the arrows and end ticks
\ifthenelse{\iii=\nlev}{
	\draw [arrowbe] (\x{1},\y{\iii}+\massrad+\storydofendtick) --
	(\x{1}+\driftdist*\useeval{\iii-1},\y{\iii}+\massrad+\storydofendtick); %story DOF arrows
	\draw (\x{1}+\driftdist*\useeval{\iii-1},\y{\iii}+\massrad+\storydofendtick/2) --
	(\x{1}+\driftdist*\useeval{\iii-1},\y{\iii}+\massrad+\storydofendtick*1.5); %Story DOF end tick
	}{
	\draw [arrowbe] (\x{1},\y{\iii}+\storydofshift) -- (\x{1}+\driftdist*\useeval{\iii-1},\y{\iii}+\storydofshift); %story DOF arrows
	\draw (\x{1}+\driftdist*\useeval{\iii-1},\y{\iii}+\storydofshift-\storydofendtick/2) --
	(\x{1}+\driftdist*\useeval{\iii-1},\y{\iii}+\storydofshift+\storydofendtick/2); %Story DOF end tick
	}

% Place the start ticks
\ifthenelse{\starttickstatus = 1}{   %Start tick for each story
	\ifthenelse{\iii=\nlev}{
		\draw (\x{1},\y{\iii}+\massrad+\storydofendtick/2) -- (\x{1},\y{\iii}+\massrad+\storydofendtick*1.5);
	}{
		\draw (\x{1},\y{\iii}+\storydofshift-\storydofendtick/2) --	(\x{1},\y{\iii}+\storydofshift+\storydofendtick/2);
	}
	}
	{\draw (0,\vertrefstart) -- (0, \y{\nlev}+\massrad+\storydofendtick*1.5);}   %Start tick is a vertical reference line for the DOFs
};
}{}

%%%%% SuperStructure DOF = 2

\ifthenelse{\showsuperstructuredof=2}{

% Place the arrows and end ticks
\ifthenelse{\doftextfloor=\nlev}{
		\draw [arrowbe] (\x{1},\y{\doftextfloor}+\massrad+\storydofendtick) --
		(\x{1}+\driftdist*\doftextstory,\y{\doftextfloor}+\massrad+\storydofendtick); %story DOF arrows
		\draw (\x{1}+\driftdist*\doftextstory,\y{\doftextfloor}+\massrad+\storydofendtick/2) --
		(\x{1}+\driftdist*\doftextstory,\y{\doftextfloor}+\massrad+\storydofendtick*1.5); %Story DOF end tick
	}
	{
		\draw [arrowbe] (\x{1},\y{\doftextfloor}+\storydofshift) -- (\x{1}+\driftdist*\doftextstory,\y{\doftextfloor}+\storydofshift); %story DOF arrows
		\draw (\x{1}+\driftdist*\doftextstory,\y{\doftextfloor}+\storydofshift-\storydofendtick/2) --
		(\x{1}+\driftdist*\doftextstory,\y{\doftextfloor}+\storydofshift+\storydofendtick/2); %Story DOF end tick
	}

% Place the start ticks
\ifthenelse{\starttickstatus = 1}{
	\ifthenelse{\doftextfloor=\nlev}
		{\draw (\x{1},\y{\doftextfloor}+\massrad+\storydofendtick/2) -- (\x{1},\y{\doftextfloor}+\massrad+\storydofendtick*1.5);}
		{\draw (\x{1},\y{\doftextfloor}+\storydofshift-\storydofendtick/2) -- (\x{1},\y{\doftextfloor}+\storydofshift+\storydofendtick/2);}
	}{
	\ifthenelse{\doftextfloor=\nlev}
		{\draw (0,\vertrefstart) -- (0, \y{\doftextfloor}+\massrad+\storydofendtick*1.5);}
		{\draw (0,\vertrefstart) -- (0, \y{\doftextfloor}+\storydofshift+\storydofendtick/2);}
	}	
}{}


%%%%%%%%%%%%%%%%%% Place the DOF Text to the selected story

% Lateral Displacement DOF
\ifthenelse{\doftextstory=\nstory}{
\path [line width=0] (\x{1},\y{\doftextfloor}+\massrad+\storydofendtick) --
node[above=-2pt, midway]{\doftext} (\x{1}+\driftdist*\doftextstory , \y{\doftextfloor}+\massrad+\storydofendtick);   %DOF text
}
{
\path [line width=0] (\x{1},\y{\doftextfloor}+\storydofshift) --
node[above=-2pt, midway]{\doftext} (\x{1}+\driftdist*\doftextstory , \y{\doftextfloor}+\storydofshift);   %DOF text
}
%Rotation DOF
\ifthenelse{\showsuperstructurerotationdof=1}{
\draw [arrowbe] (\x{1}+\driftdist*\doftextstory , \y{\doftextfloor})
++(190:\rotdofradius) arc [start angle = 190, end angle=350, radius=\rotdofradius]
node[anchor=north west, pos=\rotdoftextpos, inner sep=0pt, inner sep=0pt]
at ++(0.0cm,-0.1cm) {\rotationdoftext};
}{}

\end{scope}

%%%%%%%%%%%%%%%%%  Foundation

\ifthenelse{\showss=4}{
\fill [fill=white!75!black]
(\foundstartx, \foundboty) rectangle (\foundendx,\foundtopy);
\draw [line width = \baselinet, gray] (\foundstartx, \foundtopy) -- (\foundendx, \foundtopy);
\draw (0,\foundboty+\foundt*0.25) -- (0,0); % Vertical reference line for isolator DOF
}{}

\end{scope}
}{}

%%%%% End deflected shape


%%%%%%%%%%%%%%%%  Ground DOF

\ifthenelse{\showgrounddrift=1}{
\draw (0,\grdoftickboty) --  (0,\grdofticktopy);  %Vertical tick line for ground DOF
\draw [arrowbe] (\grdofarrstartx,\grdofarrstarty) --
node[above=-2pt, midway, xshift=1pt]{\grounddoftext}  (\grdofarrendx,\grdofarrendy);  % Ground DOF arrow
\draw (\grdofarrendx,\grdoftickboty) --  (\grdofarrendx,\grdofticktopy);  %Vertical tick line for ground DOF
}{}


%%%%%%%%%%%%%%%  Show axes

\ifthenelse{\showaxes=1}{
\draw [{->}] (\Xaxesstartx,\Xaxesstarty) -- +(\axeslenX,0) node[above]{$X$};
\draw [{->}] (\Yaxesstartx,\Yaxesstarty) -- +(0,\axeslenY) node[right]{$Y$};
}{}


\end{scope}
}  % End the macro
