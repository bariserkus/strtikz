\pgfkeys{
 /dofs/.is family, /dofs,
  default/.style={
  startx = 0cm,
  starty = 0cm,
  length x = 1cm,
  length y = 1cm,
  offset x= 0.1cm,
  offset y= 0.1cm,
  offset rotation xdir= 0.1cm,
  offset rotation ydir= 0.1cm,  
  start angle = 110,
  end angle = 340,  
  radius = 0.5cm,
  xstring = $x$,
  ystring = $y$,  
  rstring = $z$,
  show dofx = 1,
  show dofy = 1,
  show dofr = 1,
  font size = \normalsize,
  rotation = 0,
  font rotation = 0,
  font rotation for x = 0,
  font rotation for y = 0,
  font rotation for r = 0,
  dof arrow size ratio=1,
  dof arrow tip length=4pt,
  dof arrow tip width=2pt,},
  startx/.store in = \startdofx,
  starty/.store in = \startdofy,
  length x/.store in = \lengthdofx,
  length y/.store in = \lengthdofy,
  offset x/.store in = \offsetdofx,
  offset y/.store in = \offsetdofy,
  offset rotation xdir/.store in = \offsetrotx,  
  offset rotation ydir/.store in = \offsetroty, 
  start angle/.store in = \rotdofstartang,
  end angle/.store in = \rotdofendang,
  radius/.store in = \rotdofradius,
  xstring/.store in = \xdoftext,
  ystring/.store in = \ydoftext,
  rstring/.store in = \rotdoftext,
  show dofx/.store in = \showdofx,
  show dofy/.store in = \showdofy,
  show dofr/.store in = \showdofr,
  font size/.store in = \doftextsize,
  rotation/.store in = \rotationdofcenter,
  font rotation/.store in = \fontrotation,
  font rotation for x/.store in = \fontrotationx,
  font rotation for y/.store in = \fontrotationy,
  font rotation for r/.store in = \fontrotationr,      
  dof arrow size ratio/.store in = \dofarrowsizeratio,  
  dof arrow tip length/.store in = \dofarrlen,
  dof arrow tip width/.store in = \dofarrwid,}

\newcommand{\dofs}[1][]{
\pgfkeys{/dofs, default, #1}

\tikzmath{
real \startdofx, \startdofy, \lengthdofx, \lengthdofy;
real \rotdofradius, \rotdofendang, \rotdofstartang, \rotationdofcenter;
real \offsetdofx, \offsetdofy, \offsetrotx, \offsetroty;
real \fontrotation, \dofarrowsizeratio;
real \fontrotationx, \fontrotationy, \fontrotationr;
real \dofarrowsizeratio, \dofarrlen, \dofarrwid;
coordinate \cjoint; % coordinate of the joint
%Conversion all units to the unit pt
\startdofx = \startdofx;
\startdofy = \startdofy;
\lengthdofx = \lengthdofx;
\lengthdofy = \lengthdofy;
\rotdofradius = \rotdofradius;
\offsetdofx = \offsetdofx;
\offsetdofy = \offsetdofy;
\offsetrotx = \offsetrotx;
\offsetroty = \offsetroty;
\dofarrlen = \dofarrlen;
\dofarrwid = \dofarrwid;
%End conversion, All values are now in pt units.
if \rotationdofcenter==0 then {\fontrotation=\fontrotation;} else {\fontrotation=\rotationdofcenter;};
if \fontrotationx==0 then {\fontrotationx=\fontrotation;} else {\fontrotationx=\fontrotationx;};
if \fontrotationy==0 then {\fontrotationy=\fontrotation;} else {\fontrotationy=\fontrotationy;};
if \fontrotationr==0 then {\fontrotationr=\fontrotation;} else {\fontrotationr=\fontrotationr;};
}

%Settings for arrows
\tikzset{arrowdof/.style={arrows={-
>[length=\dofarrlen, width=\dofarrwid, scale=\dofarrowsizeratio, flex=1, bend]}}};
%\tikzset{reversearrowdof/.style={arrows={
%[length=\dofarrlen, width=\dofarrwid, scale=\dofarrowsizeratio, flex=1, bend]<-}}};


% Drawing everything in pt units
\begin{scope}[x=1pt, y=1pt, xshift=\startdofx, yshift=\startdofy, rotate=\rotationdofcenter]; 

%Draw x DOF
\ifthenelse{\showdofx=1}{
\draw [arrowdof] (\offsetdofx,\offsetdofy) 
-- node[at end, inner sep=1pt, outer sep=0pt, anchor=180+\fontrotationx]{\doftextsize \xdoftext}
++(\lengthdofx,0);
}{;};

\ifthenelse{\showdofx=-1}{
\draw [arrowdof] (\offsetdofx,\offsetdofy) 
-- node[at end, inner sep=1pt, outer sep=0pt, anchor=180+\fontrotationx]{\doftextsize \xdoftext}
++(-\lengthdofx,0);
}{;};

%Draw y DOF
\ifthenelse{\showdofy=1}{
\draw [arrowdof] (\offsetdofx,\offsetdofy) 
-- node[at end, inner sep=1pt, outer sep=0pt, anchor=270+\fontrotationy]{\doftextsize \ydoftext}
++(0,\lengthdofy);
}{;};

\ifthenelse{\showdofy=-1}{
\draw [arrowdof] (\offsetdofx,\offsetdofy) 
-- node[at end, inner sep=1pt, outer sep=0pt, anchor=270+\fontrotationy]{\doftextsize \ydoftext}
++(0,-\lengthdofy);
}{;};

%Draw z DOF
%\draw [arrowdof] (\offsetcenter,0) -- node[at end, inner sep=1pt, outer sep=0pt, anchor=180+\rotationdofcenter]{\doftextsize \xdoftext} ++(\lengthdofx,0);
%\draw [arrowdof] (0, \offsetcenter) -- node[at end, inner sep=1pt, outer sep=0pt, anchor=270+\rotationdofcenter]{\doftextsize \ydoftext}++(0,\lengthdofy);
\ifthenelse{\showdofr=1}{
\draw [arrowdof] (\rotdofstartang:\rotdofradius) ++(\offsetrotx,\offsetroty)
arc [start angle=\rotdofstartang, end angle=\rotdofendang, radius=\rotdofradius]
node[at end, inner sep=1pt, outer sep=0pt, anchor=\rotdofendang+\fontrotationr-90]
{\doftextsize \rotdoftext};
}{;};

\ifthenelse{\showdofr=-1}{
\draw [arrowdof] (\rotdofendang:\rotdofradius) ++(\offsetrotx,\offsetroty)
arc [start angle=\rotdofendang, end angle=\rotdofstartang, radius=\rotdofradius]
node[at end, inner sep=1pt, outer sep=0pt, anchor=\rotdofendang+\fontrotationr-90]
{\doftextsize \rotdoftext};
}{;};

\end{scope}
}