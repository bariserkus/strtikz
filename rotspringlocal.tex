\pgfkeys{
 /rotspringlocal/.is family, /rotspringlocal,
  default/.style={
  startx = 0cm,
  starty = 0cm,
  start rigid length = 2cm,
  end rigid length = 2cm,
  dof arrow ratio = 0.2,
  dof fontsize = \tiny,
  rotation = 30,
  start angle = 100,
  end angle = 260,
  radius ratio = 0.80,
  cycle number = 5,
  cycle diameter = 0.0005cm,
  rigid line thickness = 2pt,
  spring line thickness = 1pt,
  spring stiffness = $k_\theta$,
  spring mode = 1,
  spring coefficient = 0.75,
  stiffness location = above,
  stiff location offset = 6pt,
  rigid left text = {$L_1=0$},
  rigid right text = {$L_2=0$},
  rigid text offset = -2pt},
  startx/.store in =\startx,
  starty/.store in =\starty,
  start rigid length/.store in =\srl,
  end rigid length/.store in =\erl,
  dof arrow ratio/.store in = \arr,
  dof fontsize/.store in = \doffontsize,
  rotation/.store in = \anglrs,
  start angle/.store in =\startangrel,
  end angle/.store in =\endangrel,
  radius ratio/.store in =\radratio,
  cycle number/.store in = \cycleno,
  cycle diameter/.store in = \cycledia,
  rigid line thickness/.store in = \rigidlinet,
  spring line thickness/.store in = \springlinet,
  spring stiffness/.store in = \springstiff,
  spring mode/.store in =\springmode,
  spring coefficient/.store in = \springcoefficient,
  stiffness location/.store in = \stiffloc,
  stiff location offset/.store in = \stifflocoff,
  rigid left text/.store in = \xxrigidlefttext,
  rigid right text/.store in = \xxrigidrighttext,
  rigid text offset/.store in = \xxriglentextoff,  
}

\newcommand{\rotspringlocal}[1][]{
\pgfkeys{/rotspringlocal, default, #1}
\tikzmath{
real \startx, \starty, \dx, \dy, \arr;
real \anglrs, \leng, \lengdof;
real \srl, \erl;
real \localxx, \localxy, \localxdx, \localxdy;
real \tempdistadd;
real \localystart, \localyleng, \localyangl;
real \anglx, \dotrad;
real \startangi, \endangi, \startangj, \endangj, \radofdof;
real \cycleno, \cycledia;
real \stifflocoff;
real \springcoefficient;
int \springmode
%Conversion all units to the unit pt
\startx = \startx;
\starty = \starty;
\srl = \srl;
\erl = \erl;
\cycledia = \cycledia;
\stifflocoff = \stifflocoff;
%End conversion, All values are now in pt units
\leng = \srl+\erl;
\dx = cos(\anglrs)*\leng;
\dy = sin(\anglrs)*\leng;
\lengdof = \arr*\leng;
\rotrad = 0.8*\lengdof;
\localxx = (1+\arr)*\dx;
\localxy = (1+\arr)*\dy;
\localxdx = \lengdof*cos(\anglrs);
\localxdy = \lengdof*sin(\anglrs);
\tempdistadd = 0.0cm;
\tempdistadd = \tempdistadd;
\localystart = \lengdof + \tempdistadd;
\localyleng = \lengdof;
\localyangl = \anglrs+90;
\anglx = 1.5*\lengdof;
\dotrad = 0.75*\rigidlinet;
\startangi = \startangrel+\anglrs;
\endangi = \endangrel+\anglrs;
\startangj = \startangrel+\anglrs+180;
\endangj = \endangrel+\anglrs+180;
\radofdof = \radratio*\lengdof;
}
\begin{scope}[x=1pt, y=1pt, xshift=\startx, yshift=\starty, rotate=0];    % Draw everything in pt units

\rotspring[startx = 0cm, starty = 0cm, rotation = \anglrs, start rigid length = \srl, end rigid length = \erl,
rotational spring diameter = \cycledia, rotational spring cycle number = \cycleno,
text = \springstiff, text location = \stiffloc, text location offset = \stifflocoff,
rigid thickness=\rigidlinet, spring thickness=\springlinet, 
spring direction mode = \springmode, radius coefficient = \springcoefficient,
rigid left text = \xxrigidlefttext, rigid right text = \xxrigidrighttext, rigid text offset = \xxriglentextoff]

\node at (0,0)      [left=-0.05cm]{\small $i$};
\node at (\anglrs:\leng)  [right=-0.1cm]{\small $j$};
\draw [->] (\lengdof,0) -- node[at end, above]{$X$}++(\dx,0);
\draw [->] (0,\lengdof) -- node[at end, right]{$Y$}++(0,\dy);
\draw [->] (\localxx,\localxy) -- node[at end, above]{$x$}++(\localxdx,\localxdy);
\draw [->] (\localyangl:\localystart)-- node[at end, above=-2pt]{$y$}++(\localyangl:\localyleng);
\draw [arrows={->[]}](\anglx,0) arc [start angle=0, end angle=\anglrs, radius=\anglx] node[midway, right=-0.05cm]{$\theta$};
\fill [black](0,0) circle[radius=\dotrad];
\fill [black](\dx,\dy) circle[radius=\dotrad];
\draw [arrows={->[scale=0.5]}] (\startangi:\radofdof) arc [start angle=\startangi, end angle=\endangi, radius=\radofdof]
node[at end, right=-3pt]{\doffontsize 1};
\draw [arrows={->[scale=0.5]}] (\anglrs:\leng) ++(\startangj:\radofdof) arc [start angle=\startangj, end angle=\endangj, radius=\radofdof]
node[at end, left=-3pt]{\doffontsize 2};
\end{scope}
}