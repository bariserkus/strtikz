\pgfkeys{
 /rollersupportround/.is family, /rollersupportround,
  default/.style = {x value = 0,
  y value = 0,
  support width = 0.2cm,
  support height = 0.3cm,
  line width = 1.5pt,
  pin radius = 1.5pt,
  show pin = 0,
  base width = 0.8cm,
  base height = 0.2cm,
  base gray height = 0.2cm,
  base circle line width = 1.0pt,
  base circle location ratio = 0.15,
  show circles = 1,},
  x value/.store in = \xvalue,
  y value/.store in = \yvalue,
  support width/.store in = \supwidth,
  support height/.store in = \supheight,
  line width/.store in = \linewidth,
  pin radius/.store in = \pinradius,
  show pin/.store in = \showpin,
  base width/.store in = \basewidth,
  base height/.store in = \baseheight,
  base gray height/.store in = \basegrayheight,
  base circle line width/.store in = \circlinewidth,
  base circle location ratio/.store in = \circlocrat,
  show circles/.store in = \showcircles,
}

\newcommand{\rollersupportround}[1][]{
\pgfkeys{/rollersupportround, default, #1}
\tikzmath{
real \xvalue, \yvalue, \supwidth, \supheight, \linewidth, \pinradius;
real \basewidth, \baseheight, \basegrayheight, \circlinewidth, \circlocrat;
int \showpin, \showcircles;
%Conversion all units to the unit pt
\xvalue = \xvalue;
\yvalue = \yvalue;
\supwidth  = \supwidth;
\supheight  = \supheight;
\linewidth = \linewidth;
\pinradius = \pinradius;
\basewidth = \basewidth;
\baseheight = \baseheight;
\basegrayheight = \basegrayheight;
\circlinewidth = \circlinewidth;
%All units are converted to pt
\wfsx = -\supwidth/2 -\linewidth*1.02; %white fill start x
\wfsy = -\supheight; %white fill start x
\wfex = \supwidth/2 + \linewidth*1.02; %white fill end x
\wfey = -\supheight-\supwidth/2-\linewidth; %white fill end x
%
\gfsx = -\basewidth/2; %gray fill start x
\gfex = \basewidth/2; %gray fill end x
\gfsy = -\supheight-\baseheight; %gray fill start y
\gfey = -\supheight-\baseheight-\basegrayheight; %gray fill end y
%
\flsx = -\basewidth/2; % first line start x
\flex = \basewidth/2; % first line end x
\flsy = -\supheight; % first line start y
\fley = -\supheight; % first line end y
%
\slsx = \flsx; % second line start x
\slex = \flex; % second line end x
\slsy = \flsy-\baseheight; % second line start y
\sley = \fley-\baseheight; % second line end y
%
\mccx = 0; %middle circle center x
\mccy = (\flsy + \slsy) / 2; %middle circle center y
%
\lccx = -\basewidth/2 * (1-\circlocrat*2); %left circle center x
\lccy = \mccy; %left circle center y
%
\rccx = -\lccx; %right circle center x
\rccy = \lccy; %right circle center y
%
\circrad = (\flsy - \slsy) / 2 - \linewidth/3*2; %circle radius
}

\begin{scope}[x=1pt, y=1pt, xshift=\xvalue, yshift=\yvalue, rotate=0];    % Draw everything in pt units

	\draw [line width=\linewidth, double distance=\supwidth, line cap=round] (0,-\supheight) -- (0,0);
	
	\ifthenelse {\showpin = 1}{\fill (0,0) circle (\pinradius);}{};
	
	\fill [white] (\wfsx, \wfsy) rectangle (\wfex, \wfey);
	\draw [line width=\linewidth] (\flsx, \flsy) -- (\flex, \fley);
	
	\ifthenelse {\showcircles = 1}{
	\draw [line width=\circlinewidth] (\mccx, \mccy) circle (\circrad);
	\draw [line width=\circlinewidth] (\lccx, \lccy) circle (\circrad);
	\draw [line width=\circlinewidth] (\rccx, \rccy) circle (\circrad);}{};
	
	
	\fill [gray] (\gfsx, \gfsy) rectangle (\gfex, \gfey);
	\draw [line width=\linewidth] (\slsx, \slsy) -- (\slex, \sley);

\end{scope}
}