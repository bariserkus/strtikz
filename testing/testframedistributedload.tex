\documentclass[]{standalone}
%\usepackage{mathptmx}
%\renewcommand{\familydefault}{\rmdefault}
\usepackage[T1]{fontenc}
\usepackage[latin9]{inputenc}
\usepackage{siunitx}
\usepackage{array}
\usepackage{amsmath}
\usepackage{ifthen}
\usepackage{pgfplots}
\pgfplotsset{compat=1.14}
\usepackage{titling, graphicx}
\usepackage{tikz}
\usepackage{upgreek}
\usepackage{amsmath,amsthm}
\usepackage{strtikz}
\usetikzlibrary{shapes,arrows.meta,intersections,graphs,graphs.standard}
\usetikzlibrary{bending, math,fit}
\usetikzlibrary{calc,intersections,through,backgrounds,decorations.pathmorphing}
%\usetikzlibrary{fpu}
%\usepackage{pgfmath}


\begin{document}


\begin{tikzpicture}
\draw [line width=1](0 cm, 0 cm) -- (5 cm, 5 cm);
\framedistributedload[startx = 0cm,
starty = 0cm,
endx = 5cm,
endy = 5cm,
number of arrows = 10,
height of arrows = -0.5cm,
space = -0.1cm,
start ratio = 0.02,
end ratio = 0.02,
load text = $\SI{10}{\kilo\newton/\meter}$,
text shiftx=10pt,
text shifty=20pt,
arrow length=5pt,
arrow width=3pt,
arrow scale=1,
arrow line thickness=1pt,]
%$\SI{10}{\kilo\newton/\meter}$
\end{tikzpicture}


\end{document}