\documentclass[]{standalone}
%\usepackage{mathptmx}
%\renewcommand{\familydefault}{\rmdefault}
\usepackage[T1]{fontenc}
\usepackage[latin9]{inputenc}
\usepackage{siunitx}
\usepackage{array}
\usepackage{amsmath}
\usepackage{ifthen}
\usepackage{pgfplots}
\pgfplotsset{compat=1.14}
\usepackage{titling, graphicx}
\usepackage{tikz}
\usepackage{upgreek}
\usepackage{amsmath,amsthm}
\usepackage{strtikz}
\usetikzlibrary{shapes,arrows.meta,intersections,graphs,graphs.standard}
\usetikzlibrary{bending, math,fit, patterns, patterns.meta}
\usetikzlibrary{calc,intersections,through,backgrounds,decorations.pathmorphing}
%\usetikzlibrary{fpu}
%\usepackage{pgfmath}


\begin{document}


\begin{tikzpicture}
\draw [line width=1](0, 0) -- ++(10cm,5cm);
\frametrapezoidalload[startx = 0cm,
starty = 0cm,
endx = 10cm,
endy = 5cm,
number of arrows = 19,
print first arrow = 1,
print last arrow = 1,
height of arrows left= 0.5cm,
height of arrows mid = 3cm,
height of arrows right= 1cm,
mid ratio left = 0.3,
mid ratio right = 0.5,
space = 0.05cm,
start ratio = 0.02,
end ratio = 0.02,
mid load text = $\SI{20}{\kilo\newton/\meter}$,
left load text = $\SI{10}{\kilo\pascal}$,
right load text = $10kN$,
text color=blue,
mid text shiftx=0pt,
mid text shifty=-2pt,
left text shiftx=-10pt,
left text shifty=5pt,
right text shiftx=-10pt,
right text shifty=5pt,
arrow length=5pt,
arrow width=3pt,
arrow scale=1,
arrow line thickness=1pt,
arrow color=red,]
%$\SI{10}{\kilo\newton/\meter}$


\end{tikzpicture}




\end{document}
