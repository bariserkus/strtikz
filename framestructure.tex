\pgfkeys{
 /framestructure/.is family, /framestructure,
  default/.style = {
  number of storys = 5,
  number of bays = 3,
  story height = 1cm,
  bay width = 2cm,
  startX = 0cm,
  startY = 0cm,
  frame line thickness = 1pt,
  support width = 0.3cm,
  support height = 0.15cm,
  support line thickness = 1.5pt,
  show supports = 1,
  number of isolators = 4,
  isolator width = 0.3cm,
  isolator thickness = 0.2cm,
  isolator line thickness = 1.5pt,
  isolator shift = 0,
  isolator shift distance = 0.2cm,
  foundation thickness = 0.5cm,
  foundation side width = 1cm,
  mass radius = 2pt,
  show mass = 1,
  dof floor = 1,
  dof column = 2,
  arrow length ratio = 0.4,
  dof x rotation = 0,
  dof y rotation = 0,
  dof r rotation = 0,
  dof offset ratio = 0.075,
  rotation dof start angle = 120,
  rotation dof end angle = 280,
  show dof = 1,
  show axes = 1,
  subfloor number=2,
  left soil distance = 2cm,
  right soil distance = 4cm,
  left soil depth = 2cm,
  right soil depth = 2cm,
  soil below foundation = 2cm,
  left control distance x = 2cm,
  left control distance y = 2cm,
  right control distance x = 2cm,
  right control distance y = 2cm,
  axis seperation = 0.2cm,
  x axis length = 0.5cm,
  y axis length = 0.5cm,
  show piles=1,
  number of piles=3,
  pile depth=4cm,
  pile side space=0.5cm,
  pile diameter=0.5cm,
  pile line thickness=1pt,
  lateral load shift=0.5cm,
  lateral load type=2,
  top arrow length=1.5cm,
  base arrow length=0.75cm,},
  number of storys/.store in = \storynumber,
  number of bays/.store in = \baynumber,
  story height/.store in = \storyheight,
  bay width/.store in = \baywidth,
  startX/.store in = \startx,
  startY/.store in = \starty,
  frame line thickness/.store in = \framelinet,
  support width/.store in = \supportwidth,
  support height/.store in = \supportheight,
  support line thickness/.store in = \baselinet,
  show supports/.store in = \showsupports,
  number of isolators/.store in = \numberofisolators,
  isolator width/.store in = \isolationwidth,
  isolator thickness/.store in = \isolationdepth,
  isolator line thickness/.store in = \isolinet,
  isolator shift/.store in = \isoshiftyn,
  isolator shift distance/.store in = \isoshift,
  foundation thickness/.store in = \foundationdepth,
  foundation side width/.store in = \foundsidew,
  mass radius/.store in = \massrad,
  show mass/.store in = \showmass,
  dof floor/.store in = \doflocfloor,
  dof column/.store in = \dofloccolumn,
  arrow length ratio/.store in = \arrowlenratio,
  dof offset ratio/.store in = \dofoffsetratio,
  dof x rotation/.store in = \dofxrotation,
  dof y rotation/.store in = \dofyrotation,
  dof r rotation/.store in = \dofrrotation,
  rotation dof start angle/.store in = \rotdofstartangle,
  rotation dof end angle/.store in = \rotdofendangle,
  show dof/.store in = \showdof,
  show axes/.store in = \showaxes,
  subfloor number/.store in=\subfloors,
  left soil distance/.store in = \leftsoildist,
  right soil distance/.store in = \rightsoildist,
  left soil depth/.store in = \leftsoildepth,
  right soil depth/.store in = \rightsoildepth,
  soil below foundation/.store in = \soilbelowfound,
  left control distance x/.store in = \leftcontrolx,
  left control distance y/.store in = \leftcontroly,
  right control distance x/.store in = \rightcontrolx,
  right control distance y/.store in = \rightcontroly,
  axis seperation/.store in = \axisseperation,
  x axis length/.store in = \axeslenX,
  y axis length/.store in = \axeslenY,
  show piles/.store in=\showpiles,
  number of piles/.store in=\numberofpiles,
  pile depth/.store in=\piledepth,
  pile side space/.store in=\pilesidespace,
  pile diameter/.store in=\pilediameter,
  pile line thickness/.store in=\pilelinethickness,
  lateral load shift/.store in=\latloadshift,
  lateral load type/.store in=\latloadtype,
  top arrow length/.store in=\toparrlen,
  base arrow length/.store in=\basearrlen,}

\newcommand{\framestructure}[1][]{
\pgfkeys{/framestructure, default, #1}
\tikzmath{
int \storynumber, \baynumber, \columnnumber, \levelnumber;
int \nlevmo, \ncolmo, \iii, \j, \pileind, \showsupports;
real \storyheight, \baywidth, \startx, \starty, \xx, \y;
int \numberofisolators, \kiso, \isoshiftyn;
real \supportwidth, \supportheight, \isolationwidth, \isolationdepth, \isomidy;
real \foundationdepth, \massrad, \xiso, \isospace;
real \axisseperation, \framelinet, \baselinet, \isolinet;
real \rigbasestartx, \rigbaseendx, \isoboty, \isotopy, \isoshift;
real \foundboty, \foundtopy, \foundstartx, \foundendx;
int \doflocfloor, \dofloch, \dofloccolumn;
int \showaxes, \showdof, \showmass, \showsupports;
real \arrowlenratio, \minlen, \dofxx, \dofyy, \arrlen, \arrrad;
real \dofxrotation, \dofyrotation, \dofrrotation;
real \rotdofstartangle, \rotdofendangle;
int \subfloors;
real \basewallstartx, \basewallstarty, \basewallendx, \basewallendy;
real  \buildingwidth, \basewalldepth;
real \leftsoildist, \rightsoildist, \leftsoildepth, \rightsoildepth;
real \soilbelowfound, \leftcontrolx, \leftcontroly;
real \rightcontrolx, \rightcontroly;
real \rightsoilx, \rightsoily;
real \axeslenX, \axeslenY;
real \piledepth, \pilesidespace, \pilespace, \pilecoordy, \pilecoordx;
real \pilelinethickness;
real \latloadshift, \arrstartx, \toparrlen, \addtempy, \basearrlen;
int \latloadtype, \iarr;
int \substextnum;
%Conversion all units to the unit pt
\storyheight = \storyheight;
\baywidth = \baywidth;
\startx = \startx;
\starty = \starty;
\supportwidth = \supportwidth;
\supportheight = \supportheight;
\isolationwidth = \isolationwidth;
\isolationdepth = \isolationdepth;
\isoshift=\isoshift;
\foundationdepth = \foundationdepth;
\framelinet = \framelinet;
\baselinet = \baselinet;
\isolinet = \isolinet;
\massrad = \massrad;
\foundsidew = \foundsidew;
\leftsoildist = \leftsoildist;
\rightsoildist = \rightsoildist;
\leftsoildepth = \leftsoildepth;
\rightsoildepth = \rightsoildepth;
\soilbelowfound = \soilbelowfound;
\leftcontrolx = \leftcontrolx;
\leftcontroly = \leftcontroly;
\rightcontrolx = \rightcontrolx;
\rightcontroly = \rightcontroly;
\axeslenX = \axeslenX;
\axeslenY = \axeslenY;
\piledepth=\piledepth;
\pilesidespace=\pilesidespace;
\pilediameter=\pilediameter;
\pilelinethickness=\pilelinethickness;
\latloadshift=\latloadshift;
\toparrlen=\toparrlen;
\basearrlen=\basearrlen;
%End conversion, All values are now in pt units.
\columnnumber = \baynumber+1; %number of columns
\levelnumber = \storynumber+1; %number of levels
if \storynumber>1 then {\nlevmo = \levelnumber-1;} else {\nlevmo=2;};
if \baynumber>1 then {\ncolmo = \columnnumber-1;} else {\ncolmo=2;};
for \iii in {1,...,{\levelnumber}}{
\y{\iii} = (\iii-1)*\storyheight;
for \j in {1,...,{\columnnumber}}{
\x{\j} = (\j-1)*\baywidth;
};
};
\rigbasestartx = \x1-\supportwidth;
\rigbaseendx = \x{\columnnumber}+\supportwidth;
\isoboty = -\supportheight-\baselinet/2-\isolationdepth;
\isotopy = -\supportheight-\baselinet/2;
\foundboty = -\supportheight-\baselinet-\isolationdepth-\foundationdepth;
\foundtopy = -\supportheight-\baselinet-\isolationdepth;
\foundstartx = \x1-\foundsidew;
\foundendx = \x{\columnnumber}+\foundsidew;
\structheight=\nstory*\storyheight;
\isomidy = \supportheight+\isolinet+\isolationdepth/2-\baselinet;
%%%Basement%%%%
\basewallstartx=-\supportwidth;
\buildingwidth=\baynumber*\baywidth;
\basewalldepth=\subfloors*\storyheight;
\basewallstarty=\basewalldepth;
\basewallendx=\buildingwidth+\supportwidth;
\basewallendy=\basewallstarty;
\rightsoilx=\buildingwidth+\rightsoildist;
\rightsoily=\basewalldepth-\rightsoildepth;
%Some isolator properties
\isospace = (\buildingwidth)/(\numberofisolators-1);
for \kiso in {1,...,{\numberofisolators}}{
\xiso{\kiso} = (\kiso-1)*\isospace;
};
if equal(\isoshiftyn,1) then
{\xiso{1}=\xiso{1}+\isoshift;
\xiso{\numberofisolators}=\xiso{\numberofisolators}-\isoshift;}
else {};
%%%DOFs%%%%
if greater(\doflocfloor,\storynumber) then {\doflocfloor=0;} else{};
if greater(\dofloccolumn,\columnnumber) then {\dofloccolumn=1;} else{};
\dofloch = 1+\doflocfloor;
\minlen = min(\storyheight,\baywidth);
\dofxx = \x{\dofloccolumn}+\dofoffsetratio*\minlen;
\dofyy = \y{\dofloch}+\dofoffsetratio*\minlen;
\arrlen = \arrowlenratio*\minlen;
\arrrad = \arrlen*0.8;
%%%%%Axes%%%%%
\Xaxesstarty = \y{1};
\Yaxesstartx = \x{1};
\Yaxesstarty = \y{\levelnumber}+\axisseperation;
if equal(\showsupports,0) then
{\Xaxesstartx = \x{\columnnumber}+\axisseperation;} else
{\Xaxesstartx = \x{\columnnumber}+\axisseperation+\supportwidth/2;};
if equal(\showsupports,1) then
{\Xaxesstartx = \x{\columnnumber}+\axisseperation+\supportwidth/2;} else
{\Xaxesstartx = \x{\columnnumber}+\axisseperation+\supportwidth;};
%%%%Piles%%%%
\pilecoordy = -\supportheight;
\pilespace = (2*\supportwidth+\buildingwidth-2*\pilesidespace)/(\numberofpiles-1);
for \pileind in {1,...,{\numberofpiles}}{
\pilecoordx{\pileind} = -\supportwidth+\pilesidespace+(\pileind-1)*\pilespace;
};
%%%%Lateral Loads%%%%
if or(equal(\latloadtype,1), equal(\latloadtype,3)) then {
for \iarr in {1,...,{\levelnumber}}{
	\arrstartx{\iarr}=-\toparrlen*\y{\iarr}/\structheight-\latloadshift;};
} else {};
if equal(\latloadtype,2) then {
	\arrstartx{0}=-\latloadshift;
	\y{0}=-\isomidy;
	for \iarr in {1,...,{\levelnumber}}{
		if equal(\iarr,1) then{\addtempy=\supportheight/2;} else {\addtempy=0;};
		\arrstartx{\iarr}=-\toparrlen*(\y{\iarr}+\isomidy-\addtempy)/
		(\structheight+\isomidy)-\latloadshift;
	};
} else {};
}   %End tikzmath


\begin{scope}[x=1pt, y=1pt, xshift=\startx, yshift=\starty, rotate=0];
	% Drawing everything in pt units


\ifthenelse{\latloadtype=1}{
	\foreach \iarr in {2,...,{\levelnumber}}{
		\tikzmath{\substextnum=\iarr-1;}
		\draw[arrows={-[length=5pt, width=4pt]>}, line width=1pt]
		(\arrstartx{\iarr},\y{\iarr}) --
		node[anchor=south, pos=0.4, inner sep=1pt]{$F_{\substextnum}$}
		(-\latloadshift,\y{\iarr});}
	\draw (\arrstartx{1},\y{1}) -- (\arrstartx{\levelnumber},\y{\levelnumber});
	\draw (\arrstartx{1},\y{1}) -- (\arrstartx{1},\y{\levelnumber});
}{;}

\ifthenelse{\latloadtype=2}{
	\foreach \iarr in {1,...,{\levelnumber}}{
		\tikzmath{\substextnum=\iarr-1;}
		\ifthenelse{\iarr=1}{
			\draw[arrows={-[length=2pt, width=2pt]>}, line width=0.5pt]
			(\arrstartx{\iarr},\y{\iarr}-\supportheight/2) --
			node[anchor=east, pos=0, inner sep=1pt]{$F_{\text{b}}$}
			(-\latloadshift,\y{\iarr}-\supportheight/2);
		}{
			\draw[arrows={-[length=5pt, width=4pt]>}, line width=1pt]
			(\arrstartx{\iarr},\y{\iarr}) --
			node[anchor=south, pos=0.4, inner sep=1pt]{$F_{\substextnum}$}
			(-\latloadshift,\y{\iarr});
		}
	}
	\draw (\arrstartx{0},\y{0}) -- (\arrstartx{\levelnumber},\y{\levelnumber});
	\draw (\arrstartx{0},\y{0}) -- (\arrstartx{0},\y{\levelnumber});
}{;}

\ifthenelse{\latloadtype=3}{
	\foreach \iarr in {2,...,{\levelnumber}}{
		\tikzmath{\substextnum=\iarr-1;}
		\draw[arrows={-[length=5pt, width=4pt]>}, line width=1pt]
		(\arrstartx{\iarr},\y{\iarr}) --
		node[anchor=south, pos=0.4, inner sep=1pt]{$F_{\substextnum}$}
		(-\latloadshift,\y{\iarr});}
	\draw (\arrstartx{1},\y{1}) -- (\arrstartx{\levelnumber},\y{\levelnumber});
	\draw (\arrstartx{1},\y{1}) -- (\arrstartx{1},\y{\levelnumber});
	\draw[arrows={-[length=5pt, width=4pt]>}, line width=1pt]
	(-\latloadshift-1cm,-\supportheight/2)--
	node[anchor=south, pos=0.4, inner sep=1pt]{$F_{\text{b}}$}
	++(1cm,0);
}{;}




\ifthenelse{\showsupports=5}{

\path[fill=lightgray] decorate[decoration={random steps, segment length=0.25cm}]
{(-\leftsoildist,\basewalldepth) -- ++(0,-\leftsoildepth)
..controls++(0,-\leftcontroly) and ++(-\leftcontrolx,0)..(0,-\soilbelowfound)
 -- ++(\buildingwidth,0) ..controls ++(\rightcontrolx,0) and
 ++(0,-\rightcontroly) .. (\rightsoilx,\rightsoily)
 -- ++(0,\rightsoildepth)}
-- cycle;

\draw(-\leftsoildist,\basewalldepth) --
(\buildingwidth+\rightsoildist,\basewalldepth);

\path[fill=white](0,\basewalldepth) -- ++(\buildingwidth,0) --
++(0,-\basewalldepth) -- ++(-\buildingwidth,0) -- cycle;

\ifthenelse{\showpiles=1}{
\foreach \pileind in {1,...,{\numberofpiles}}
{\pilesupport[x coordinate=\pilecoordx{\pileind}, y coordinate=\pilecoordy,
pile depth=\piledepth, pile diameter=\pilediameter,
pile line thickness=\pilelinethickness, fill color=gray,]}
}{}

\filldraw[line width = \baselinet, fill=gray]
(\basewallstartx,\basewallstarty) -- ++(\supportwidth,0) -- ++(0,-\basewalldepth)
-- ++(\buildingwidth,0) -- ++(0,\basewalldepth) -- ++(\supportwidth,0) --
++(0,-\basewalldepth-\supportheight) -- (\basewallstartx,-\supportheight) -- cycle;

}{}


%\draw (0,0)-- (0.5cm,2cm);
\draw [line width = \framelinet]
(\x{1},\y{1}) -- (\x{1}, \y{\levelnumber}) --
(\x{\columnnumber}, \y{\levelnumber}) -- (\x{\columnnumber}, \y{1});
\foreach \iii in {2,...,{\nlevmo}}
{\draw [line width = \framelinet] (\x{1},\y{\iii}) -- (\x{\columnnumber},\y{\iii});}

\foreach \j in {2,...,{\ncolmo}}
{\draw [line width = \framelinet] (\x{\j},\y{1}) -- (\x{\j},\y{\levelnumber});}


\ifthenelse{\showmass=1}{
\ifthenelse{\showsupports=5}{
	\foreach \iii in {\useeval{2+\subfloors},...,{\levelnumber}}
	{\foreach \j in {1,...,{\columnnumber}}{
	\shade[ball color=black] (\x{\j},\y{\iii}) circle (\massrad);
	}}
	\foreach \iii in {2,...,\useeval{\subfloors+1}}
	{\foreach \j in {2,...,\useeval{\columnnumber-1}}{
	\shade[ball color=black] (\x{\j},\y{\iii}) circle (\massrad);
	}}
}
{
	\foreach \iii in {2,...,{\levelnumber}}
	{\foreach \j in {1,...,{\columnnumber}}{
	\shade[ball color=black] (\x{\j},\y{\iii}) circle (\massrad);
	}}
}
}{}

\ifthenelse{\showsupports=0}{
}{}

\ifthenelse{\showsupports=1}{
\foreach \j in {1,...,{\columnnumber}}
{
\fill [gray] (\x{\j},\y{1}) +({-\supportwidth/2},-\supportheight)
rectangle +({\supportwidth/2},\y{1});
\draw [line width = \baselinet]
(\x{\j},\y{1}) +({-\supportwidth/2},\y{1}) -- +({\supportwidth/2},\y{1});
}
}{}

\ifthenelse{\showsupports=2}{
\fill [gray] (\rigbasestartx, 0) rectangle (\rigbaseendx,-\supportheight);
\draw [line width = \baselinet] (\rigbasestartx, 0) -- (\rigbaseendx,0);
}{}

\ifthenelse{\showsupports=3}{
\foreach \j in {1,...,{\columnnumber}}
{\fill [fill=gray, draw=black, line width = \baselinet] (\rigbasestartx, 0) rectangle
	(\rigbaseendx,-\supportheight);}
}{}

\ifthenelse{\showsupports=4}{
\fill [fill=gray, draw=black, line width = \baselinet] (\rigbasestartx, 0) rectangle
(\rigbaseendx,-\supportheight);

\foreach \j in {1,...,{\numberofisolators}}
{\filldraw [fill = white!20!black, draw=black, line width=\isolinet]
(\xiso{\j},\y{1}) +(-\isolationwidth/2,\isoboty) rectangle
+(\isolationwidth/2,\isotopy);}

\fill [fill=gray]
(\foundstartx, \foundboty) rectangle (\foundendx,\foundtopy);
\draw [line width = \baselinet] (\foundstartx, \foundtopy) -- (\foundendx, \foundtopy);
}{}

\ifthenelse{\showaxes=1}{
\draw [{->}] (\Xaxesstartx,\Xaxesstarty) -- +(\axeslenX,0) node[above]{$X$};
\draw [{->}] (\Yaxesstartx,\Yaxesstarty) -- +(0,\axeslenY) node[right]{$Y$};
}{}

\ifthenelse{\showdof=1}{
\dofs[startx = \dofxx, starty = \dofyy, length x = \arrlen,  length y = \arrlen,
  offset x= 0.0cm, offset y= 0.0cm,
  offset rotation xdir= 0cm, offset rotation ydir= 0cm,
  start angle = \rotdofstartangle, end angle = \rotdofendangle, radius = \arrrad,
  xstring = $i$, ystring = $i+1$, rstring = $i+2$, font size = \tiny,
  rotation=0, font rotation=0,
  font rotation for x = \dofxrotation, font rotation for y = \dofyrotation,
  font rotation for r = \dofrrotation,
  arrow size ratio=0.5,
  ]  
}{}


\end{scope}
}

