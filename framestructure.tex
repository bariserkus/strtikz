\pgfkeys{
 /framestructure/.is family, /framestructure,
  default/.style = {%
  number of storys = 5,
  show open story=0,
  left story=0,
  right story=0,
  number of bays = 5,
  show less bays=0,
  left bays=0,
  right bays=0,
  story height = 1cm,
  bay width = 2cm,
  startX = 0cm,
  startY = 0cm,
  line thickness = 1.5pt,
  beam line thickness = 1.5pt,
  ground beam line thickness = 1.5pt,
  beam color=blue,
  column line thickness = 1.5pt,
  column color=black,
  show shear wall=1,
  shear wall bay=2,
  shear wall line thickness=0.75pt,
  shear wall fill color=gray,
  support width = 0.3cm,
  support height = 0.15cm,
  support line thickness = 1.5pt,
  support line color=black,
  support fill color=gray!65,
  show supports = 1,
  number of isolators = 4,
  isolator width = 0.3cm,
  isolator thickness = 0.2cm,
  isolator line thickness = 1.5pt,
  isolator line color=black,
  isolator fill color=gray,
  isolator shift = 0,
  isolator shift distance = 0.2cm,
  isolator floor line thickness=1pt,
  isolator floor line color=black,
  isolator floor fill color=gray,
  isolator deflection ratio=1.0,
  foundation thickness = 0.5cm,
  foundation side width = 1cm,
  show mass = 1,
  mass radius = 2pt,
  mass color super=black,
  mass color sub=red,
  show dof = 1,
  show dofx = 1,
  show dofy = 1,
  show dofr = 1,
  dof xstring = $x$,
  dof ystring = $y$,  
  dof rstring = $\theta$,
  dof floor = 1,
  dof column = 2,
  arrow length ratio = 0.4,
  rot arrow length ratio = 0.4,
  dof x rotation = 0,
  dof y rotation = 0,
  dof r rotation = 0,
  dof offset ratio = 0.075,
  rotation dof start angle = 120,
  rotation dof end angle = 280,
  dofx pos = 1,
  dofy pos = 1,
  dofr pos = 1,
  dofx in sep=1pt,
  dofy in sep=1pt,
  dofr in sep=1pt,
  dof offset x= 0.1cm,
  dof offset y= 0.1cm,
  dof offset rot x= 0.1cm,
  dof offset rot y= 0.1cm, 
  dof arrow ratio=0.75,
  dof text ratio=0.5,
  show axes = 1,
  subfloor number=0,
  half space type=1,
  half space fill color=brown!75!white,
  support space left=0cm,
  support space right=0cm,
  support space bottom=0cm,
  half space line thickness=1pt,
  top line thickness=0.5pt,
  left soil distance = 2cm,
  right soil distance = 4cm,
  left soil depth = 2cm,
  right soil depth = 2cm,
  soil below foundation = 2cm,
  left control distance x = 2cm,
  left control distance y = 2cm,
  right control distance x = 2cm,
  right control distance y = 2cm,
  axis seperation = 0.2cm,
  x axis length = 0.5cm,
  y axis length = 0.5cm,
  show piles=1,
  number of piles=3,
  pile depth=4cm,
  pile side space=0.5cm,
  pile diameter=0.5cm,
  pile line thickness=1pt,
  pile fill color=gray,
  show pile boundary line=0,
  pile boundary line thickness=2pt,
  pile boundary line color=blue,
  pile boundary line type=dashed,
  show lateral load = 0,
  lateral load shift=0.5cm,
  lateral load type=2,
  top arrow length=1.5cm,
  base arrow length=0.75cm,
  show deflection = 0, 
  interstory drift = 0.2cm,
  defl parameter x = 0.2cm,
  defl parameter y = 0.5cm,
  defl parameter base = 0.15cm,
  show eng bedrock=0,
  eng bedrock depth=2cm,
  eng bedrock left distance=1cm,
  eng bedrock right distance=1cm,
  eng bedrock line thickness=1.5pt,
  eng bedrock text=Bedrock,
  eng bedrock fill color=gray,
  show trans layer=0,
  trans layer line thickness=2pt,
  trans layer color=red,
  trans layer line type=dashed,
  show ss interface=0,
  ss interface line thickness=3pt,
  ss interface color=green,
  ss interface line type=dashed,
  show substructure mark=0,
  substructure mark space=0.30cm,
  substructure mark radius=0.25cm,
  substructure mark line thickness=0.75pt,
  substructure mark line type=dashed,
  substructure mark line color=blue,
  show superstructure shade=0,
  superstructure shade space=5pt,
  superstructure shade opacity=0.75,
  show superstructure mark=0,
  superstructure mark space=0.30cm,
  superstructure mark radius=0.25cm,
  superstructure mark line thickness=0.75pt,
  superstructure mark line type=dashed,
  superstructure mark line color=red,
  phorizontal show springs=1,
  phorizontal number of springs = 5,
  phorizontal spring direction = 1,
  phorizontal space between springs = 0.15cm,
  phorizontal space = 0.25cm,
  phorizontal start ratio = 0.05,
  phorizontal end ratio = 0.05,
  phorizontal spring text = {},
  phorizontal text color = black,
  phorizontal text shiftx=0pt,
  phorizontal text shifty=0pt,
  phorizontal spring prelength ratio=0.2,
  phorizontal spring postlength ratio=0.2,
  phorizontal spring segment=0.15cm,
  phorizontal spring width=3pt,
  phorizontal spring scale=1,
  phorizontal spring line thickness=1pt,
  phorizontal spring color = black,
  phorizontal support width=0.3cm,
  phorizontal support depth=0.1cm,
  phorizontal support line thickness=1pt,
  phorizontal show support shade=1,
  phorizontal support shade color=gray,
  pvertical show springs=1,
  pvertical number of springs = 5,
  pvertical space = 0.0cm,
  pvertical start ratio = 0.1,
  pvertical end ratio = 0.1,
  pvertical spring text = {},
  pvertical text color = black,
  pvertical text shiftx=0pt,
  pvertical text shifty=0pt,
  pvertical spring length=0.5cm,
  pvertical spring prelength ratio=0.15,
  pvertical spring postlength ratio=0.15,
  pvertical spring width=0.1cm,
  pvertical spring segment=0.15cm,
  pvertical spring scale=1,
  pvertical spring line thickness=1pt,
  pvertical spring color=black,
  pvertical support width=0.3cm,
  pvertical support depth = 0.1cm,
  pvertical support line thickness = 1pt,
  pvertical show support shade=1,
  pvertical support shade color=gray,
  paxial show spring=1,
  paxial spring text = {},
  paxial text color = black,
  paxial text shiftx=0pt,
  paxial text shifty=0pt,
  paxial spring length=0.5cm,
  paxial spring prelenratio=0.25,
  paxial spring postlength ratio=0.1,
  paxial spring segment=0.15cm,
  paxial spring width=0.1cm,
  paxial spring line thickness=1pt,
  paxial spring color=black,
  paxial support width=0.3cm,
  paxial support depth = 0.1cm,
  paxial support line thickness = 1pt,
  paxial show support shade=1,
  paxial support shade color=gray,
  left wall show springs=0,
  left wall number of springs = 5,
  left wall spring space = 0.0cm,
  left wall spring start ratio = 0.1,
  left wall spring end ratio = 0.1,
  left wall spring text = {},
  left wall spring text shiftx=0pt,
  left wall spring text location=above,
  left wall spring text shifty=0pt,
  left wall spring length=0.5cm,
  left wall spring prelength ratio=0.2,
  left wall spring postlength ratio=0.1,
  left wall spring width=0.07cm,
  left wall spring segment=0.15cm,
  left wall spring scale=1,
  left wall spring line thickness=1pt,
  left wall spring color=black,
  left wall spring support width=0.3cm,
  left wall spring support depth = 0.1cm,
  left wall spring support line thickness = 1pt,
  left wall spring support show shade=1,
  left wall spring support shade color=gray,
  right wall show springs=0,
  right wall number of springs = 5,
  right wall spring space = 0.0cm,
  right wall spring start ratio = 0.1,
  right wall spring end ratio = 0.1,
  right wall spring text = {},
  right wall spring text shiftx=0pt,
  right wall spring text shifty=0pt,
  right wall spring text location=above,
  right wall spring length=0.5cm,
  right wall spring prelength ratio=0.2,
  right wall spring postlength ratio=0.1,
  right wall spring width=0.07cm,
  right wall spring segment=0.15cm,
  right wall spring scale=1,
  right wall spring line thickness=1pt,
  right wall spring color=black,
  right wall spring support width=0.3cm,
  right wall spring support depth = 0.1cm,
  right wall spring support line thickness = 1pt,
  right wall spring support show shade=1,
  right wall spring support shade color=gray,
  foundation show springs=0,
  foundation number of springs = 30,
  foundation spring space = 0.0cm,
  foundation spring start ratio = 0.05,
  foundation spring end ratio = 0.05,
  foundation spring text = {},
  foundation spring text shiftx=0pt,
  foundation spring text shifty=0pt,
  foundation spring text location=above,
  foundation spring length=0.5cm,
  foundation spring prelength ratio=0.2,
  foundation spring postlength ratio=0.1,
  foundation spring width=0.07cm,
  foundation spring segment=0.15cm,
  foundation spring scale=1,
  foundation spring line thickness=1pt,
  foundation spring color=black,
  foundation spring support width=0.3cm,
  foundation spring support depth = 0.1cm,
  foundation spring support line thickness = 1pt,
  foundation spring support show shade=1,
  foundation spring support shade color=gray,
  show left pin supports=1,
  show right pin supports=1,
  show bottom pin supports=1,
  left pin support triangle width=0.2cm,
  left pin support show shade=1,
  left pin support shade color=gray,
  left pin support line width=0.4cm,
  left pin support line depth=0.2cm,
  left pin support line thickness=1pt,
  left pin support radius ratio=0.2,
  left pin support x value=0,
  left pin support y value=0,
  left pin support show circle=0,
  left pin support rotation=-90,
  right pin support triangle width=0.2cm,
  right pin support shade color=gray,
  right pin support show shade=1,
  right pin support line width=0.4cm,
  right pin support line depth=0.2cm,
  right pin support line thickness=1pt,
  right pin support radius ratio=0.2,
  right pin support x value=0,
  right pin support y value=0,
  right pin support show circle=0,
  right pin support rotation=90,
  bottom pin support number=10,
  bottom pin support triangle width=0.2cm,
  bottom pin support show shade=1,
  bottom pin support shade color=gray,
  bottom pin support line width=0.4cm,
  bottom pin support line depth=0.2cm,
  bottom pin support line thickness=1pt,
  bottom pin support radius ratio=0.2,
  bottom pin support x value=0,
  bottom pin support y value=0,
  bottom pin support show circle=0,
  bottom pin support rotation=0,},
  number of storys/.store in=\storynumber,
  show open story/.store in=\showopenstory,
  left story/.store in=\leftstory,
  right story/.store in=\rightstory,
  number of bays/.store in = \baynumber,
  show less bays/.store in=\showlessbays,
  left bays/.store in=\leftbays,
  right bays/.store in=\rightbays,
  story height/.store in = \storyheight,
  bay width/.store in = \baywidth,
  startX/.store in = \startx,
  startY/.store in = \starty,
  line thickness/.store in = \linet,
  beam line thickness/.store in=\beamlinet,
  ground beam line thickness/.store in=\groundbeamlinet,
  beam color/.store in=\beamcolor,
  column line thickness/.store in=\collinet,
  column color/.store in=\columncolor,
  show shear wall/.store in=\showshearwall,
  shear wall bay/.store in=\shearwallbay,
  shear wall line thickness/.store in=\shearwalllinet,
  shear wall fill color/.store in=\shearwallfillcolor,
  support width/.store in = \supportwidth,
  support height/.store in = \supportheight,
  support line thickness/.store in = \baselinet,
  support line color/.store in = \supportlinecolor,
  support fill color/.store in=\supportfillcolor,
  show supports/.store in = \showsupports,
  number of isolators/.store in = \numberofisolators,
  isolator width/.store in = \isolationwidth,
  isolator thickness/.store in = \isolationdepth,
  isolator line thickness/.store in = \isolinet,
  isolator line color/.store in=\isolinecolor,
  isolator fill color/.store in=\isofillcolor,
  isolator shift/.store in = \isoshiftyn,
  isolator shift distance/.store in = \isoshift,
  isolator deflection ratio/.store in=\isodeflratio,
  isolator floor line thickness/.store in=\isofloorlinethk,
  isolator floor line color/.store in=\isofloorlinecolor,
  isolator floor fill color/.store in=\isofloorfillcolor,
  foundation thickness/.store in = \foundationdepth,
  foundation side width/.store in = \foundsidew,
  show mass/.store in = \showmass,
  mass radius/.store in = \massrad,
  mass color super/.store in = \masscolorsuper,
  mass color sub/.store in = \masscolorsub,
  show dof/.store in = \showdof,
  show dofx/.store in = \shodofx,
  show dofy/.store in = \shodofy,
  show dofr/.store in = \shodofr,
  dof floor/.store in = \doflocfloor,
  dof xstring/.store in = \dofxstr,
  dof ystring/.store in = \dofystr,
  dof rstring/.store in = \dofrstr,
  dof column/.store in = \dofloccolumn,
  arrow length ratio/.store in = \arrowlenratio,
  rot arrow length ratio/.store in = \rotarrowlenratio,
  dof offset ratio/.store in = \dofoffsetratio,
  dof x rotation/.store in = \dofxrotation,
  dof y rotation/.store in = \dofyrotation,
  dof r rotation/.store in = \dofrrotation,
  rotation dof start angle/.store in = \rotdofstartangle,
  rotation dof end angle/.store in = \rotdofendangle,
  dofx pos/.store in = \dofposx,
  dofy pos/.store in = \dofposy,
  dofr pos/.store in = \dofposr,
  dofx in sep/.store in = \dofinnersepx,
  dofy in sep/.store in = \dofinnersepy,
  dofr in sep/.store in = \dofinnersepr,
  dof offset x/.store in = \dofoffsetx,
  dof offset y/.store in = \dofoffsety,
  dof offset rot x/.store in = \dofrotoffsetx,  
  dof offset rot y/.store in = \dofrotoffsety,
  dof arrow ratio/.store in = \dofarrowratio,
  dof text ratio/.store in = \doftextratio,
  show axes/.store in = \showaxes,
  subfloor number/.store in=\subfloors,
  half space type/.store in=\halfspacetype,
  half space fill color/.store in=\halfspacefillcolor,
  support space left/.store in=\hsdistleft,
  support space right/.store in=\hsdistright,
  support space bottom/.store in=\hsdistbottom,
  half space line thickness/.store in=\hsinnerlinethk,
  top line thickness/.store in=\thinlinethk,
  left soil distance/.store in = \leftsoildist,
  right soil distance/.store in = \rightsoildist,
  left soil depth/.store in = \leftsoildepth,
  right soil depth/.store in = \rightsoildepth,
  soil below foundation/.store in = \soilbelowfound,
  left control distance x/.store in = \leftcontrolx,
  left control distance y/.store in = \leftcontroly,
  right control distance x/.store in = \rightcontrolx,
  right control distance y/.store in = \rightcontroly,
  axis seperation/.store in = \axisseperation,
  x axis length/.store in = \axeslenX,
  y axis length/.store in = \axeslenY,
  show piles/.store in=\showpiles,
  number of piles/.store in=\numberofpiles,
  pile depth/.store in=\piledepth,
  pile side space/.store in=\pilesidespace,
  pile diameter/.store in=\pilediameter,
  pile line thickness/.store in=\pilelinethickness,
  pile fill color/.store in=\pilefillcolor,
  show pile boundary line/.store in=\showpbline,
  pile boundary line thickness/.store in=\pblinet,
  pile boundary line color/.store in=\pblinecolor,
  pile boundary line type/.store in=\pblinetype,
  show lateral load/.store in=\showlatload,
  lateral load shift/.store in=\latloadshift,
  lateral load type/.store in=\latloadtype,
  top arrow length/.store in=\toparrlen,
  base arrow length/.store in=\basearrlen,
  show deflection/.store in=\showdefl, 
  interstory drift/.store in=\drift,
  defl parameter x/.store in=\defparx,
  defl parameter y/.store in=\defpary,
  defl parameter base/.store in=\defbase,
  show eng bedrock/.store in=\showengbedrock,
  eng bedrock depth/.store in=\engbedrockdepth,
  eng bedrock left distance/.store in=\engbedrockleftdist,
  eng bedrock right distance/.store in=\engbedrockrightdist,
  eng bedrock line thickness/.store in=\engbedrocklinewidth,
  eng bedrock text/.store in=\engbedrocktext,
  eng bedrock fill color/.store in=\engbedrockfillcolor,
  show trans layer/.store in=\showtranslayer,
  trans layer line thickness/.store in= \translayerlinet,
  trans layer color/.store in=\translayercolor,
  trans layer line type/.store in=\translayerlinetype,
  show ss interface/.store in=\showssinter,
  ss interface line thickness/.store in=\ssinterlinet,
  ss interface color/.store in=\ssintercolor,
  ss interface line type/.store in=\ssinterlinetype,
  show substructure mark/.store in=\showmarkss,
  substructure mark space/.store in=\markssexspace,
  substructure mark radius/.store in=\markssrad,
  substructure mark line thickness/.store in=\marksslinet,
  substructure mark line type/.store in=\marksslinetype,
  substructure mark line color/.store in=\marksslinecolor,
  show superstructure shade/.store in=\showsupershade,
  superstructure shade space/.store in=\supershadespace,
  superstructure shade opacity/.store in=\supershadeopacity,
  show superstructure mark/.store in=\showmarksuper,
  superstructure mark space/.store in=\marksuperexspace,
  superstructure mark radius/.store in=\marksuperrad,
  superstructure mark line thickness/.store in=\marksuperlinet,
  superstructure mark line type/.store in=\marksuperlinetype,
  superstructure mark line color/.store in=\marksuperlinecolor,
  phorizontal show springs/.store in = \showphorizontalsprings,
  phorizontal number of springs/.store in = \phorspringnumber,
  phorizontal spring direction/.store in = \phorspringdir,
  phorizontal space between springs/.store in = \phorspbtwspr,
  phorizontal space/.store in = \phorspringspace,
  phorizontal start ratio/.store in = \phorspringstartratio,
  phorizontal end ratio/.store in = \phorspringendratio,
  phorizontal spring text/.store in = \phorspringtext,
  phorizontal text color/.store in = \phortextcolor,
  phorizontal text shiftx/.store in = \phortextshiftx,
  phorizontal text shifty/.store in = \phortextshifty,
  phorizontal spring prelength ratio/.store in = \phorprelenratio,
  phorizontal spring postlength ratio/.store in = \phorpostlenratio,
  phorizontal spring segment/.store in = \phorsegm,
  phorizontal spring width/.store in = \phorsprwid,
  phorizontal spring scale/.store in = \phorsprscale,
  phorizontal spring line thickness/.store in = \phorsprlinethk,
  phorizontal spring color/.store in = \phorsprcolor,
  phorizontal support width/.store in = \phorsuppwidth,
  phorizontal support depth/.store in = \phorsuppdepth,
  phorizontal support line thickness/.store in = \phorsupplinethk,
  phorizontal show support shade/.store in = \phorshowsuppshade,
  phorizontal support shade color/.store in = \phorsuppshadecol,
  pvertical show springs/.store in = \showpverticalsprings,
  pvertical number of springs/.store in = \pverspringnumber,
  pvertical space/.store in = \pverspringspace,
  pvertical start ratio/.store in = \pverspringstartratio,
  pvertical end ratio/.store in = \pverspringendratio,
  pvertical spring text/.store in = \pverspringtext,
  pvertical text color/.store in = \pvertextcolor,
  pvertical text shiftx/.store in = \pvertextshiftx,
  pvertical text shifty/.store in = \pvertextshifty,
  pvertical spring length/.store in = \pverspringlength,
  pvertical spring prelength ratio/.store in = \pverprelenratio,
  pvertical spring postlength ratio/.store in = \pverpostlenratio,
  pvertical spring width/.store in = \pverampl,
  pvertical spring segment/.store in = \pversegm,
  pvertical spring scale/.store in = \pverspringscale,
  pvertical spring line thickness/.store in = \pverspringthk,
  pvertical spring color/.store in = \pverspringcolor,
  pvertical support width/.store in = \pversuppwidth,
  pvertical support depth/.store in = \pversuppdepth,
  pvertical support line thickness/.store in = \pversupplinethk,
  pvertical show support shade/.store in = \pvershowsuppshade,
  pvertical support shade color/.store in = \pversuppshadecol,
  paxial show spring/.store in = \showpaxialspring,
  paxial spring text/.store in = \paxialspringtext,
  paxial text color/.store in = \paxialtextcolor,
  paxial text shiftx/.store in = \paxialtextshiftx,
  paxial text shifty/.store in = \paxialtextshifty,
  paxial spring length/.store in=\paxialspringlength,
  paxial spring prelenratio/.store in=\paxialspringprelenratio,
  paxial spring postlength ratio/.store in=\paxialspringpostlenratio,
  paxial spring segment/.store in=\paxialspringsegm,
  paxial spring width/.store in=\paxialspringwidth,
  paxial spring line thickness/.store in=\paxialspringlinethk,
  paxial spring color/.store in=\paxialspringcolor,
  paxial support width/.store in = \paxialsuppwidth,
  paxial support depth/.store in = \paxialsuppdepth,
  paxial support line thickness/.store in = \paxialsupplinethk,
  paxial show support shade/.store in = \paxialshowsuppshade,
  paxial support shade color/.store in = \paxialsuppshadecol,
  left wall show springs/.store in=\leftwshowsprings,
  left wall number of springs/.store in =\leftwspringnumber,
  left wall spring space/.store in =\leftwspringspace,
  left wall spring start ratio/.store in =\leftwspringstartratio,
  left wall spring end ratio/.store in =\leftwspringendratio,
  left wall spring text/.store in =\leftwspringtext,
  left wall spring text shiftx/.store in =\leftwtextshiftx,
  left wall spring text shifty/.store in =\leftwtextshifty,
  left wall spring text location/.store in=\leftwspringtextloc,
  left wall spring length/.store in =\leftwspringlength,
  left wall spring prelength ratio/.store in =\leftwprelenratio,
  left wall spring postlength ratio/.store in =\leftwpostlenratio,
  left wall spring width/.store in =\leftwampl,
  left wall spring segment/.store in =\leftwsegm,
  left wall spring scale/.store in =\leftwspringscale,
  left wall spring line thickness/.store in =\leftwspringthk,
  left wall spring color/.store in =\leftwspringcolor,
  left wall spring support width/.store in =\leftwsuppwidth,
  left wall spring support depth/.store in =\leftwsuppdepth,
  left wall spring support line thickness/.store in =\leftwsupplinethk,
  left wall spring support show shade/.store in =\leftwshowsuppshade,
  left wall spring support shade color/.store in =\leftwsuppshadecol,
  right wall show springs/.store in=\rightwshowsprings,
  right wall number of springs/.store in =\rightwspringnumber,
  right wall spring space/.store in =\rightwspringspace,
  right wall spring start ratio/.store in =\rightwspringstartratio,
  right wall spring end ratio/.store in =\rightwspringendratio,
  right wall spring text/.store in =\rightwspringtext,
  right wall spring text shiftx/.store in =\rightwtextshiftx,
  right wall spring text shifty/.store in =\rightwtextshifty,
  right wall spring text location/.store in=\rightwspringtextloc,
  right wall spring length/.store in =\rightwspringlength,
  right wall spring prelength ratio/.store in =\rightwprelenratio,
  right wall spring postlength ratio/.store in =\rightwpostlenratio,
  right wall spring width/.store in =\rightwampl,
  right wall spring segment/.store in =\rightwsegm,
  right wall spring scale/.store in =\rightwspringscale,
  right wall spring line thickness/.store in =\rightwspringthk,
  right wall spring color/.store in =\rightwspringcolor,
  right wall spring support width/.store in =\rightwsuppwidth,
  right wall spring support depth/.store in =\rightwsuppdepth,
  right wall spring support line thickness/.store in =\rightwsupplinethk,
  right wall spring support show shade/.store in =\rightwshowsuppshade,
  right wall spring support shade color/.store in =\rightwsuppshadecol,
  foundation show springs/.store in=\foundshowsprings,
  foundation number of springs/.store in =\foundspringnumber,
  foundation spring space/.store in =\foundspringspace,
  foundation spring start ratio/.store in =\foundspringstartratio,
  foundation spring end ratio/.store in =\foundspringendratio,
  foundation spring text/.store in =\foundspringtext,
  foundation spring text shiftx/.store in =\foundtextshiftx,
  foundation spring text shifty/.store in =\foundtextshifty,
  foundation spring text location/.store in=\foundspringtextloc,
  foundation spring length/.store in =\foundspringlength,
  foundation spring prelength ratio/.store in =\foundprelenratio,
  foundation spring postlength ratio/.store in =\foundpostlenratio,
  foundation spring width/.store in =\foundampl,
  foundation spring segment/.store in =\foundsegm,
  foundation spring scale/.store in =\foundspringscale,
  foundation spring line thickness/.store in =\foundspringthk,
  foundation spring color/.store in =\foundspringcolor,
  foundation spring support width/.store in =\foundsuppwidth,
  foundation spring support depth/.store in =\foundsuppdepth,
  foundation spring support line thickness/.store in =\foundsupplinethk,
  foundation spring support show shade/.store in =\foundshowsuppshade,
  foundation spring support shade color/.store in =\foundsuppshadecol,
  show left pin supports/.store in=\showleftpinsup,
  show right pin supports/.store in=\showrightpinsup,
  show bottom pin supports/.store in=\showbottompinsup,
left pin support triangle width/.store in=\lefttriw,
left pin support show shade/.store in=\showleftpinsuppshade,
left pin support shade color/.store in=\leftpinsuppshadecolor,
left pin support line width/.store in=\leftlinew,
left pin support line depth/.store in=\leftlined,
left pin support line thickness/.store in=\leftlinet,
left pin support radius ratio/.store in=\leftradratio,
left pin support x value/.store in=\leftxvalue,
left pin support y value/.store in=\leftyvalue,
left pin support show circle/.store in=\leftshowcircle,
left pin support rotation/.store in=\leftrotpin,
right pin support triangle width/.store in=\righttriw,
right pin support show shade/.store in=\showrightpinsuppshade,
right pin support shade color/.store in=\rightpinsuppshadecolor,
right pin support line width/.store in=\rightlinew,
right pin support line depth/.store in=\rightlined,
right pin support line thickness/.store in=\rightlinet,
right pin support radius ratio/.store in=\rightradratio,
right pin support x value/.store in=\rightxvalue,
right pin support y value/.store in=\rightyvalue,
right pin support show circle/.store in=\rightshowcircle,
right pin support rotation/.store in=\rightrotpin,
bottom pin support number/.store in=\nbotpin,
bottom pin support show shade/.store in=\showbottompinsuppshade,
bottom pin support shade color/.store in=\bottompinsuppshadecolor,
bottom pin support triangle width/.store in=\bottomtriw,
bottom pin support line width/.store in=\bottomlinew,
bottom pin support line depth/.store in=\bottomlined,
bottom pin support line thickness/.store in=\bottomlinet,
bottom pin support radius ratio/.store in=\bottomradratio,
bottom pin support x value/.store in=\bottomxvalue,
bottom pin support y value/.store in=\bottomyvalue,
bottom pin support show circle/.store in=\bottomshowcircle,
bottom pin support rotation/.store in=\bottomrotpin,}
\newcommand{\framestructure}[1][]{
\pgfkeys{/framestructure, default, #1}
\tikzmath{
int \storynumber, \baynumber, \columnnumber, \levelnumber, \storyminone;
int \nlevmo, \ncolmo, \iii, \j, \pileind;
real \storyheight, \baywidth, \startx, \starty, \xx, \y;
int \numberofisolators, \kiso, \isoshiftyn;
real \supportwidth, \supportheight, \isolationwidth, \isolationdepth, \isomidy;
real \foundationdepth, \massrad, \xiso, \isospace;
real \axisseperation, \linet, \baselinet, \isolinet;
real \beamlinet, \collinet;
real \rigbasestartx, \rigbaseendx, \isoboty, \isotopy, \isoshift;
real \isodeflratio, \isodeflx;
real \foundboty, \foundtopy, \foundstartx, \foundendx;
int \doflocfloor, \dofloch, \dofloccolumn;
int \showaxes, \showdof, \showmass, \showsupports;
real \arrowlenratio, \minlen, \dofxx, \dofyy, \arrlen, \arrrad;
real \dofxrotation, \dofyrotation, \dofrrotation;
real \rotdofstartangle, \rotdofendangle;
int \subfloors, \deflstart, \deflstartplusone, \superstorynumber;
real \basewallstartx, \basewallstarty, \basewallendx, \basewallendy;
real \buildingwidth, \basewalldepth;
int \halfspacetype;
real \leftsoildist, \rightsoildist, \leftsoildepth, \rightsoildepth;
real \soilbelowfound, \leftcontrolx, \leftcontroly;
real \rightcontrolx, \rightcontroly;
real \rightsoilx, \rightsoily;
real \axeslenX, \axeslenY;
real \piledepth, \pilesidespace, \pilespace, \pilecoordy, \pilecoordx;
real \pilelinethickness, \pblinet;
int \showpbline;
real \latloadshift, \arrstartx, \toparrlen, \addtempy, \basearrlen;
int \latloadtype, \iarr, \showlatload;
int \substextnum;
int \showdefl;
int \istory, \ibay, \icol, \ibeam, \idefl, \jdefl;
real \drift, \deflect, \defparx, \defpary, \defparbeamx, \defparbeamy, \defbase;
real \tempdim, \fixbeamx, \fixbeamy, \fixcolx, \fixcoly;
int \dofcounterx, \dofcountery, \dofcounterz;
int \showengbedrock;
real \engbedrockdepth, \engbedrockleftdist, \engbedrockrightdist;
real \engbedrockstartx, \engbedrockstarty;
real \engbedrockdeltax, \engbedrockdeltay;
real \engbedrocklinewidth;
int \showtranslayer, \showssinter;
real \translayerleftstartx, \translayerleftstarty, \translayerleftdeltay;
real \translayerrightstartx, \translayerrightstarty, \translayerrightdeltay;
real \translayerlinet, \ssinterlinet;
int \showmarkss;
real \markssleftstartx, \markssleftstarty;
real \markssdeltax, \markssdeltay, \markssrad, \marksslinet, \markssexspace;
real \supershadespace;
real \supershadestartx, \supershadestarty, \supershadedeltax, \supershadedeltay;
real \supershadespace, \supershadeopacity;
int \showsupershade;
int \showlessbays, \leftbays, \rightbays, \startcol, \endcol, \startbeam, \endbeam;
int \showshearwall, \shearwallbay, \shearwallstartcolumn, \shearwallendcolumn;
real \shearwallstartx, \shearwallstarty \shearwalldeltax, \shearwalldeltay;
real \shearwalllinet;
int \showmarksuper;
real \marksuperexspace, \marksuperrad, \marksuperlinet;
real \marksuperleftstartx, \marksuperleftstarty,
real \marksuperdeltax, \marksuperdeltay;
int \leftstory, \leftlevel, \rightstory, \rightlevel;
int \leftopenstory, \rightopenstory, \showopenstory;
real \leftwallh, \leftopenh, \rightwallh, \rightopenh, \diffwallh;
int \showphorizontalsprings, \phorspringnumber, \phorspringdir;
real \phorspbtwspr, \phorspringspace, \phorspringstartratio, \phorspringendratio;
real \phortextshiftx, \phortextshifty, \phorprelenratio, \phorpostlenratio;
real \phorsegm, \phorsprwid, \phorsprscale, \phorsprlinethk;
real \phorsuppwidth, \phorsuppdepth, \phorsupplinethk,
int \phorshowsuppshade;
int \showpverticalsprings, \pverspringnumber;
real \pverspringspace, \pverspringstartratio, \pverspringendratio,
real \pvertextshiftx, \pvertextshifty, \pverspringlength, \pverprelenratio, \pverpostlenratio;
real \pverampl, \pversegm, \pverspringscale, \pverspringthk;
real \pversuppwidth, \pversuppdepth, \pversupplinethk;
int \pvershowsuppshade;
int \showpaxialspring;
real \pvertextshiftx, \pvertextshifty;
real \paxialspringlength, \paxialspringprelenratio, \paxialspringpostlenratio;
real \paxialspringsegm, \paxialspringwidth, \paxialspringlinethk;
real \paxialsuppwidth, \paxialsuppdepth, \paxialsupplinethk;
int \paxialshowsuppshade;
int \leftwshowsprings, \leftwspringnumber;
real \leftwspringspace, \leftwspringstartratio, \leftwspringendratio;
real \leftwtextshiftx, \leftwtextshifty, \leftwspringlength,
real \leftwprelenratio, \leftwpostlenratio, \leftwampl, \leftwsegm;
real \leftwspringscale, \leftwspringthk, \leftwsuppwidth;
real \leftwsuppdepth, \leftwsupplinethk;
int \leftwshowsuppshade;
int \rightwshowsprings, \rightwspringnumber;
real \rightwspringspace, \rightwspringstartratio, \rightwspringendratio;
real \rightwtextshiftx, \rightwtextshifty, \rightwspringlength,
real \rightwprelenratio, \rightwpostlenratio, \rightwampl, \rightwsegm;
real \rightwspringscale, \rightwspringthk, \rightwsuppwidth;
real \rightwsuppdepth, \rightwsupplinethk;
int \rightwshowsuppshade;
int \foundshowsprings, \foundspringnumber;
real \foundspringspace, \foundspringstartratio, \foundspringendratio;
real \foundtextshiftx, \foundtextshifty, \foundspringlength,
real \foundprelenratio, \foundpostlenratio, \foundampl, \foundsegm;
real \foundspringscale, \foundspringthk, \foundsuppwidth;
real \foundsuppdepth, \foundsupplinethk;
int \foundshowsuppshade;
real \pinnedy, \aaa;
int \showleftpinsup, \showrightpinsup, \showbottompinsup;
int \nbotpin, \ibot, \ijkk;
real \botpinx, \botpinspace, \bottomlength;
real \lefttriw, \leftlinew, \leftlined, \leftlinet, \leftradratio;
real \leftxvalue, \leftyvalue;
int \leftshowcircle;
real \leftrotpin;
real \righttriw, \rightlinew, \rightlined, \rightlinet, \rightradratio;
real \rightxvalue, \rightyvalue;
int \rightshowcircle;
real \rightrotpin;
real \bottomtriw, \bottomlinew, \bottomlined, \bottomlinet, \bottomradratio;
real \bottomxvalue, \bottomyvalue;
int \bottomshowcircle;
real \bottomrotpin;
real \hsdistleft, \hsdistright, \hsdistbottom, \hsinnerlinethk, \thinlinethk;
%Conversion all units to the unit pt
\hsdistleft=\hsdistleft;
\hsdistright=\hsdistright;
\hsdistbottom=\hsdistbottom;
\hsinnerlinethk=\hsinnerlinethk;
\thinlinethk=\thinlinethk;
\hsdistleft=\hsdistleft;
\hsdistright=\hsdistright;
\hsdistbottom=\hsdistbottom;
\hsinnerlinethk=\hsinnerlinethk;
\storyheight = \storyheight;
\baywidth = \baywidth;
\startx = \startx;
\starty = \starty;
\supportwidth = \supportwidth;
\supportheight = \supportheight;
\isolationwidth = \isolationwidth;
\isolationdepth = \isolationdepth;
\isoshift=\isoshift;
\foundationdepth = \foundationdepth;
\linet = \linet;
\beamlinet = \beamlinet;
\collinet = \collinet;
\baselinet = \baselinet;
\isolinet = \isolinet;
\massrad = \massrad;
\foundsidew = \foundsidew;
\leftsoildist = \leftsoildist;
\rightsoildist = \rightsoildist;
\leftsoildepth = \leftsoildepth;
\rightsoildepth = \rightsoildepth;
\soilbelowfound = \soilbelowfound;
\leftcontrolx = \leftcontrolx;
\leftcontroly = \leftcontroly;
\rightcontrolx = \rightcontrolx;
\rightcontroly = \rightcontroly;
\axeslenX = \axeslenX;
\axeslenY = \axeslenY;
\piledepth=\piledepth;
\pilesidespace=\pilesidespace;
\pilediameter=\pilediameter;
\pilelinethickness=\pilelinethickness;
\pblinet=\pblinet;
\latloadshift=\latloadshift;
\toparrlen=\toparrlen;
\basearrlen=\basearrlen;
\drift=\drift;
\defparx=\defparx;
\defpary=\defpary;
\defbase=\defbase;
\engbedrockdepth=\engbedrockdepth;
\engbedrockleftdist=\engbedrockleftdist;
\engbedrockrightdist=\engbedrockrightdist;
\engbedrocklinewidth=\engbedrocklinewidth;
\translayerlinet=\translayerlinet;
\ssinterlinet=\ssinterlinet;
\markssexspace=\markssexspace;
\markssrad=\markssrad;
\marksslinet=\marksslinet;
\supershadespace=\supershadespace;
\shearwalllinet=\shearwalllinet;
\marksuperexspace=\marksuperexspace;
\marksuperrad=\marksuperrad;
\marksuperlinet=\marksuperlinet;
\phorspbtwspr=\phorspbtwspr;
\phorspringspace=\phorspringspace;
\phortextshiftx=\phortextshiftx;
\phortextshifty=\phortextshifty;
\phorsegm=\phorsegm;
\phorsprwid=\phorsprwid;
\phorsprlinethk=\phorsprlinethk;
\phorsuppwidth=\phorsuppwidth;
\phorsuppdepth=\phorsuppdepth;
\phorsupplinethk=\phorsupplinethk;
\pverspringspace=\pverspringspace;
\pvertextshiftx=\pvertextshiftx;
\pvertextshifty=\pvertextshifty;
\pverspringlength=\pverspringlength;
\pverampl=\pverampl;
\pversegm=\pversegm;
\pverspringthk=\pverspringthk;
\pversuppwidth=\pversuppwidth;
\pversuppdepth=\pversuppdepth;
\pversupplinethk=\pversupplinethk;
\pvertextshiftx=\pvertextshiftx;
\pvertextshifty=\pvertextshifty;
\paxialspringlength=\paxialspringlength;
\paxialspringsegm=\paxialspringsegm;
\paxialspringwidth=\paxialspringwidth;
\paxialspringlinethk=\paxialspringlinethk;
\paxialsuppwidth=\paxialsuppwidth;
\paxialsuppdepth=\paxialsuppdepth;
\paxialsupplinethk=\paxialsupplinethk;
\leftwspringspace=\leftwspringspace;
\leftwtextshiftx=\leftwtextshiftx;
\leftwtextshifty=\leftwtextshifty;
\leftwspringlength=\leftwspringlength;
\leftwampl=\leftwampl;
\leftwsegm=\leftwsegm;
\leftwspringthk=\leftwspringthk;
\leftwsuppwidth=\leftwsuppwidth;
\leftwsuppdepth=\leftwsuppdepth;
\leftwsupplinethk=\leftwsupplinethk;
\rightwspringspace=\rightwspringspace;
\rightwtextshiftx=\rightwtextshiftx;
\rightwtextshifty=\rightwtextshifty;
\rightwspringlength=\rightwspringlength;
\rightwampl=\rightwampl;
\rightwsegm=\rightwsegm;
\rightwspringthk=\rightwspringthk;
\rightwsuppwidth=\rightwsuppwidth;
\rightwsuppdepth=\rightwsuppdepth;
\rightwsupplinethk=\rightwsupplinethk;
\foundspringspace=\foundspringspace;
\foundtextshiftx=\foundtextshiftx;
\foundtextshifty=\foundtextshifty;
\foundspringlength=\foundspringlength;
\foundampl=\foundampl;
\foundsegm=\foundsegm;
\foundspringthk=\foundspringthk;
\foundsuppwidth=\foundsuppwidth;
\foundsuppdepth=\foundsuppdepth;
\foundsupplinethk=\foundsupplinethk;
%\pinnedy=1cm;
%\pinnedy=\pinnedy;
\lefttriw=\lefttriw;
\leftlinew=\leftlinew;
\leftlined=\leftlined;
\leftlinet=\leftlinet;
\leftxvalue=\leftxvalue;
\leftyvalue=\leftyvalue;
\righttriw=\righttriw;
\rightlinew=\rightlinew;
\rightlined=\rightlined;
\rightlinet=\rightlinet;
\rightxvalue=\rightxvalue;
\rightyvalue=\rightyvalue;
\bottomtriw=\bottomtriw;
\bottomlinew=\bottomlinew;
\bottomlined=\bottomlined;
\bottomlinet=\bottomlinet;
\bottomxvalue=\bottomxvalue;
\bottomyvalue=\bottomyvalue;
%End conversion, All values are now in pt units.
if greater(2*\massrad,\collinet) then {
\latloadshift=\latloadshift+\massrad;} else {
\latloadshift=\latloadshift+\collinet/2;};
%%
\dofcounter=0;
%
if notequal(\showsupports,5) then {\subfloors=0;};
%
\storyminone = \storynumber-1;
\columnnumber = \baynumber+1; %number of columns
\levelnumber = \storynumber+1; %number of levels
if \storynumber>1 then {\nlevmo = \levelnumber-1;} else {\nlevmo=2;};
if \baynumber>1 then {\ncolmo = \columnnumber-1;} else {\ncolmo=2;};
%% Undeflected shape coordinates
for \iii in {1,...,{\levelnumber}}{
\y{\iii} = (\iii-1)*\storyheight;
for \j in {1,...,{\columnnumber}}{
\x{\j} = (\j-1)*\baywidth;
};
};
%% Deflected shape coordinates
if equal(\showsupports,5) then
{\deflstart=\subfloors+1;
\superstorynumber=\storynumber-\subfloors;}
else
{\deflstart=1;
\superstorynumber=\storynumber;};
\deflstartplusone=\deflstart+1;
for \idefl in {1,...,{\deflstart}}{
	\deflect{\idefl} = 0;
};
for \idefl in {\deflstart+1,...,{\levelnumber}}{
	\deflect{\idefl} = (\idefl-\deflstart)*\drift;
};
\tempdim = sqrt(\defparx*\defparx + \defpary*\defpary);
\fixbeamx = \collinet/2 * (\defpary / \tempdim);
\fixbeamy = \collinet/2 * (\defparx / \tempdim);
\fixcoly = 1.01*\beamlinet/2 * (\defpary / \tempdim);
\fixcolx = 1.01*\beamlinet/2 * (\defparx / \tempdim);
\defparbeamx = \defparx / \storyheight * \baywidth;
\defparbeamy = \defpary / \storyheight * \baywidth;
\rigbasestartx = \x1-\supportwidth;
\rigbaseendx = \x{\columnnumber}+\supportwidth;
\isoboty = -\supportheight-\baselinet/2-\isolationdepth;
\isotopy = -\supportheight-\baselinet/2;
\isodeflx=\isodeflratio*\isolationdepth*\showdefl;
\foundboty = -\supportheight-\baselinet-\isolationdepth-\foundationdepth;
\foundtopy = -\supportheight-\baselinet-\isolationdepth;
\foundstartx = \x1-\foundsidew-\isodeflx;
\foundendx = \x{\columnnumber}+\foundsidew-\isodeflx;
\structheight=\storynumber*\storyheight;
\isomidy = \supportheight+\isolinet+\isolationdepth/2-\baselinet;
%%%Unbalanced Soil
%\showopenstory=0;
%\leftstory=1;
%\rightstory=1;
if equal(\subfloors,0) then {\showopenstory=0;};
if greater(\leftstory, \subfloors) then {\leftstory=\subfloors;};
if greater(\rightstory, \subfloors) then {\rightstory=\subfloors;};
if equal(\showopenstory,0) then
{\leftstory=\subfloors;
\rightstory=\subfloors;};
\leftopenstory=\subfloors-\leftstory;
\leftlevel=\leftstory+1;
\leftwallh=\leftstory*\storyheight;
\leftopenh=\leftopenstory*\storyheight;
\rightopenstory=\subfloors-\rightstory;
\rightlevel=\rightstory+1;
\rightwallh=\rightstory*\storyheight;
\rightopenh=\rightopenstory*\storyheight;
\diffwallh=\leftwallh-\rightwallh;
%%%Basement%%%%
\soilbelowfoundtof = \soilbelowfound+\supportheight+\hsdistbottom;
\basewallstartx=-\supportwidth;
\buildingwidth=\baynumber*\baywidth;
\basewalldepth=\subfloors*\storyheight;
\basewallstarty=\basewalldepth;
\basewallendx=\buildingwidth+\supportwidth;
\basewallendy=\basewallstarty;
\rightsoilx=\hsdistright+\supportwidth+\buildingwidth+\rightsoildist;
\rightsoily=\rightwallh-\rightsoildepth-\supportheight-\hsdistbottom;
%Some isolator properties
\isospace = (\buildingwidth)/(\numberofisolators-1);
for \kiso in {1,...,{\numberofisolators}}{
\xiso{\kiso} = (\kiso-1)*\isospace;
};
if equal(\isoshiftyn,1) then
{\xiso{1}=\xiso{1}+\isoshift;
\xiso{\numberofisolators}=\xiso{\numberofisolators}-\isoshift;}
else {};
%%%DOFs%%%%
if greater(\doflocfloor,\storynumber) then {\doflocfloor=0;} else{};
if greater(\dofloccolumn,\columnnumber) then {\dofloccolumn=1;} else{};
\dofloch = 1+\doflocfloor;
\minlen = min(\storyheight,\baywidth);
\dofxx = \x{\dofloccolumn}+\dofoffsetratio*\minlen+\deflect{\dofloch}*\showdefl;
\dofyy = \y{\dofloch}+\dofoffsetratio*\minlen;
\arrlen = \arrowlenratio*\minlen;
\arrrad = \rotarrowlenratio*\minlen;
%\arrrad = \arrlen*0.8;
%%%%%Axes%%%%%
\Xaxesstarty = \y{1};
\Yaxesstartx = \x{1};
\Yaxesstarty = \y{\levelnumber}+\axisseperation;
if equal(\showsupports,0) then
{\Xaxesstartx = \x{\columnnumber}+\axisseperation;} else
{\Xaxesstartx = \x{\columnnumber}+\axisseperation+\supportwidth/2;};
if equal(\showsupports,1) then
{\Xaxesstartx = \x{\columnnumber}+\axisseperation+\supportwidth/2;} else
{\Xaxesstartx = \x{\columnnumber}+\axisseperation+\supportwidth;};
%%%%Piles%%%%
\pilecoordy = -\supportheight;
\pilespace = 
(2*\supportwidth+\buildingwidth-2*\pilesidespace)/(\numberofpiles-1);
for \pileind in {1,...,{\numberofpiles}}{
\pilecoordx{\pileind} = -\supportwidth+\pilesidespace+(\pileind-1)*\pilespace;
};
%%%%Lateral Loads%%%%
if or(equal(\latloadtype,1), equal(\latloadtype,3)) then {
for \iarr in {1,...,{\levelnumber}}{
	\arrstartx{\iarr}=-\toparrlen*\y{\iarr}/\structheight-\latloadshift;};
} else {};
if equal(\latloadtype,2) then {
	\arrstartx{0}=-\latloadshift;
	\y{0}=-\isomidy;
	for \iarr in {1,...,{\levelnumber}}{
		if equal(\iarr,1) then{\addtempy=\supportheight/2;} else {\addtempy=0;};
		\arrstartx{\iarr}=-\toparrlen*(\y{\iarr}+\isomidy-\addtempy)/
		(\structheight+\isomidy)-\latloadshift;
	};
} else {};
if equal(\latloadtype,4) then {
	\arrstartx{0}=-\latloadshift;
	\y{0}=-\isomidy;
	for \iarr in {1,...,{\levelnumber}}{
		if equal(\iarr,1) then{\addtempy=\supportheight/2;} else {\addtempy=0;};
		\arrstartx{\iarr}=-(\toparrlen-\basearrlen)*(\y{\iarr}+\isomidy-\addtempy)/
		(\structheight+\isomidy)-\latloadshift-\basearrlen;
	};
} else {};
%%%Engineering Bedrock Coordinates
%\engbedrocklinewidth=2pt;
%\engbedrocktext="Bedrock";
\engbedrockstartx=-\leftsoildist-\engbedrockleftdist;
\engbedrockstarty=-\soilbelowfoundtof-\engbedrockdepth;
\engbedrockdeltax=\leftsoildist+\engbedrockleftdist+
  \buildingwidth+\rightsoildist+\engbedrockrightdist;
\engbedrockdeltay=\engbedrockdepth;
\engbedrockmidx=\engbedrockstartx + \engbedrockdeltax/2;
\engbedrockmidy=\engbedrockstarty + \engbedrockdeltay/2;
%%%Transmitting Layer
%\translayerlinet=2pt;
\translayerlinet=\showtranslayer*\translayerlinet;
\translayerleftstartx=-\supportwidth-\hsdistleft-\leftsoildist-\translayerlinet/2;
\translayerleftstarty=\leftwallh;
\translayerleftdeltay=\leftwallh+\soilbelowfoundtof;
\translayerrightstartx=\supportwidth+\hsdistright+
	\rightsoildist+\buildingwidth+\translayerlinet/2;
\translayerrightstarty=\rightwallh;
\translayerrightdeltay=\rightwallh+\soilbelowfoundtof;
%%%Soil-Structure Interface
%\ssinterlinet=3pt;
\ssinterleftstartx=-\supportwidth-\baselinet/2-\ssinterlinet/2;
\ssinterrightstartx=\buildingwidth+\supportwidth+\baselinet/2+\ssinterlinet/2;
\ssinterleftstarty=\leftwallh;
\ssinterrightstarty=\rightwallh;
\ssinterdeltax=\ssinterrightstartx-\ssinterleftstartx;
\ssinterleftdeltay=\leftwallh+\supportheight+\baselinet/2+\ssinterlinet/2;
\ssinterrightdeltay=\rightwallh+\supportheight+\baselinet/2+\ssinterlinet/2;
%%%Mark for Substructure
%\markssexspace=0.3cm;
%\markssrad=0.25cm;
%\marksslinet=0.75pt;
\markssleftstartx=-\supportwidth-\markssexspace;
\markssleftstarty=\basewalldepth+\markssexspace;
\markssdeltax=\buildingwidth+2*\supportwidth+2*\markssexspace;
\markssdeltay=\basewalldepth+\supportheight+2*\markssexspace;
%%%Shade for Superstructure
%\supershadespace=5pt;
%\supershadeopacity=0.75;
\supershadestartx=-\collinet/2-\supershadespace;
\supershadestarty=\basewalldepth+\baselinet/2;
\supershadedeltax=\buildingwidth+\collinet+2*\supershadespace
 +\deflect{\levelnumber}*\showdefl;
\supershadedeltay=\superstorynumber*\storyheight+\beamlinet/2+\supershadespace;
%%% Superstructure with less bays
if equal(\showlessbays,0) then {
\leftbays=0;
\rightbays=0;};
\startcol=\leftbays+1;
\endcol=\columnnumber-\rightbays;
\startbeam=\leftbays+1;
\endbeam=\baynumber-\rightbays;
\superbays=\baynumber-\leftbays-\rightbays;
\superwidth=\superbays*\baywidth;
%%% Modification for the ground beam
\groundbeamshifty=\groundbeamlinet/2-\baselinet/2;
%%% Shear Wall
%\showshearwall=1;
%\shearwallbay=3;
if equal(\showdefl,1) then {\showshearwall=0;};
\shearwallstartcolumn=\shearwallbay;
\shearwallendcolumn=\shearwallbay+1;
\shearwallstartx=\x{\shearwallstartcolumn}-\collinet/2;
\shearwallstarty=\y{1}+\baselinet/2-+\shearwalllinet/2;
\shearwalldeltax=\baywidth+\collinet;
\shearwallsupportstartx=\shearwallstartx-\supportwidth/2;
\shearwallsupportdeltax=\shearwalldeltax+\supportwidth;
if equal(\superstorynumber,0) then {
\shearwalldeltay=\y{\levelnumber};} else
{\shearwalldeltay=\y{\levelnumber}-\shearwallstarty
+\beamlinet/2-\shearwalllinet/2;};
%%%Mark for Superstructure
%\showmarksuper=1;
%\marksuperexspace=0.3cm;
%\marksuperrad=0.25cm;
%\marksuperlinet=0.75pt;
\marksuperleftstartx=\x{\startcol}-\marksuperexspace;
\marksuperleftstarty=\basewalldepth+\marksuperexspace/2;
\marksuperdeltax=\superwidth+2*\marksuperexspace+\deflect{\levelnumber}*\showdefl;
\marksuperdeltay=\y{\levelnumber}-\y{\deflstart}+\marksuperexspace/2;
%%% Right lateral soil springs
\rightwallspringstartx=\buildingwidth+\supportwidth;
\rightwallspringstarty=\rightwallh;
\rightwallspringendx=\buildingwidth+\supportwidth;
\rightwallspringendy=-\supportheight;
%%% Bottom pin supports
%\nbotpin=10;
\botpinx{1}=-\supportwidth/2;
\bottomlength=\buildingwidth+\supportwidth;
\botpinspace=\bottomlength/(\nbotpin-1);
for \ibot in {2,...,{\nbotpin}}{
	\botpinx{\ibot}=\botpinx{1}+(\ibot-1)*\botpinspace;};
%%%
}   %End tikzmath




\begin{scope}[x=1pt, y=1pt, xshift=\startx, yshift=\starty, rotate=0];
	% Drawing everyg in pt units


%%%% Lateral Laods
\ifthenelse{\showlatload=1}{
\ifthenelse{\latloadtype=1}{
	\foreach \iarr in {2,...,{\levelnumber}}{
		\tikzmath{\substextnum=\iarr-1;}
		\draw[arrows={-[length=5pt, width=4pt]>}, line width=1pt]
		(\arrstartx{\iarr},\y{\iarr}) --
		node[anchor=south, pos=0.4, inner sep=1pt]{$F_{\substextnum}$}
		(-\latloadshift,\y{\iarr});}
	\draw (\arrstartx{1},\y{1}) -- (\arrstartx{\levelnumber},\y{\levelnumber});
	\draw (\arrstartx{1},\y{1}) -- (\arrstartx{1},\y{\levelnumber});
}{;}

\ifthenelse{\latloadtype=2}{
	\foreach \iarr in {1,...,{\levelnumber}}{
		\tikzmath{\substextnum=\iarr-1;}
		\ifthenelse{\iarr=1}{
			\draw[arrows={-[length=2pt, width=2pt]>}, line width=0.5pt]
			(\arrstartx{\iarr},\y{\iarr}-\supportheight/2) --
			node[anchor=east, pos=0, inner sep=1pt]{$F_{\text{b}}$}
			(-\latloadshift,\y{\iarr}-\supportheight/2);
		}{
			\draw[arrows={-[length=5pt, width=4pt]>}, line width=1pt]
			(\arrstartx{\iarr},\y{\iarr}) --
			node[anchor=south, pos=0.4, inner sep=1pt]{$F_{\substextnum}$}
			(-\latloadshift,\y{\iarr});
		}
	}
	\draw (\arrstartx{0},\y{0}) -- (\arrstartx{\levelnumber},\y{\levelnumber});
	\draw (\arrstartx{0},\y{0}) -- (\arrstartx{0},\y{\levelnumber});
}{;}

\ifthenelse{\latloadtype=4}{
	\foreach \iarr in {1,...,{\levelnumber}}{
		\tikzmath{\substextnum=\iarr-1;}
		\ifthenelse{\iarr=1}{
			\draw[arrows={-[length=5pt, width=4pt]>}, line width=1pt]
			(\arrstartx{\iarr},\y{\iarr}-\supportheight/2) --
			node[anchor=south, pos=0.4, inner sep=1pt]{$F_{\text{b}}$}
			(-\latloadshift,\y{\iarr}-\supportheight/2);
		}{
			\draw[arrows={-[length=5pt, width=4pt]>}, line width=1pt]
			(\arrstartx{\iarr},\y{\iarr}) --
			node[anchor=south, pos=0.4, inner sep=1pt]{$F_{\substextnum}$}
			(-\latloadshift,\y{\iarr});
		}
	}
	\draw (\arrstartx{0},\isoboty) -- (\arrstartx{1},\y{1}-\supportheight/2)
	-- (\arrstartx{\levelnumber},\y{\levelnumber});
	\draw (\arrstartx{0},\isoboty) -- (\arrstartx{0},\y{\levelnumber});
}{;}


\ifthenelse{\latloadtype=3}{
	\foreach \iarr in {2,...,{\levelnumber}}{
		\tikzmath{\substextnum=\iarr-1;}
		\draw[arrows={-[length=5pt, width=4pt]>}, line width=1pt]
		(\arrstartx{\iarr},\y{\iarr}) --
		node[anchor=south, pos=0.4, inner sep=1pt]{$F_{\substextnum}$}
		(-\latloadshift,\y{\iarr});}
	\draw (\arrstartx{1},\y{1}) -- (\arrstartx{\levelnumber},\y{\levelnumber});
	\draw (\arrstartx{1},\y{1}) -- (\arrstartx{1},\y{\levelnumber});
	\draw[arrows={-[length=5pt, width=4pt]>}, line width=1pt]
	(-\latloadshift,-\supportheight/2)--
	node[anchor=south, pos=0.4, inner sep=1pt]{$F_{\text{b}}$}
	++(1cm,0);
}{;}

\ifthenelse{\latloadtype=5}
{
	\foreach \istory in {\deflstartplusone,...,{\levelnumber}}
	{
		\tikzmath{\substextnum=\istory-\deflstart;}
		\draw [arrows={-[length=5pt, width=4pt]>}, line width=1pt, red]
		(\x{\startcol}+\deflect{\istory}*\showdefl-\toparrlen,\y{\istory}) --
			node[anchor=south, pos=0.4, inner sep=1pt]{$F_{\substextnum}$}		
		(\x{\startcol}+\deflect{\istory}*\showdefl-\latloadshift, \y{\istory});
	}
}{;}
}{;}


%%5. Pile Supports
\ifthenelse{\showsupports=5}{
	
	%%% Rectangular half space
	\ifthenelse{\halfspacetype=1}{
		\path[fill=\halfspacefillcolor]
		{(-\supportwidth,0) ++(-\hsdistleft,0) ++(-\leftsoildist,\leftwallh)
			-- ++(0,-\leftwallh) -- ++(0,-\soilbelowfoundtof) -- ++(\buildingwidth,0)
			-- ++(\leftsoildist,0) -- ++(\rightsoildist,0)
			-- ++(\hsdistleft,0) -- ++(\hsdistright,0)
			-- ++(\supportwidth,0) -- ++(\supportwidth,0)
			-- ++(0,\soilbelowfoundtof) -- ++(0,\rightwallh) 
			-- ++(-\rightsoildist,0) -- ++(0,-\rightwallh)
			-- ++(0,-\hsdistbottom)-- ++(0,-\supportheight)
			-- ++(-\buildingwidth,0)
			-- ++(-\hsdistleft,0) -- ++(-\hsdistright,0)
			-- ++(-\supportwidth,0) -- ++(-\supportwidth,0)
			-- ++(0,\leftwallh)-- ++(0,\hsdistbottom)} -- ++(0,\supportheight)
		-- cycle;
		
		\draw[line width=\hsinnerlinethk]
		(-\supportwidth,\thinlinethk/2) ++(-\hsdistleft,\leftwallh)
		--++(0,-\leftwallh)	--++(0,-\hsdistbottom) --++(0,-\supportheight)
		--++(0,-\thinlinethk/2) --++(\buildingwidth,0)
		--++(\hsdistleft,0)  --++(\hsdistright,0)
		-- ++(\supportwidth,0) -- ++(\supportwidth,0)
		--++(0,\thinlinethk/2) --++(0,\hsdistbottom) --++(0,\supportheight)
		--++(0,\rightwallh);
		%\draw(-5cm,-\soilbelowfoundtof) --++(15cm,0);
		
		}{}
	
	%%% Transmitting Layers
	\ifthenelse{\showtranslayer=1}{
		\draw[line width=\translayerlinet, \translayercolor, \translayerlinetype]
		(\translayerleftstartx,\translayerleftstarty)
		-- ++(0,-\translayerleftdeltay);
		
		\draw[line width=\translayerlinet, \translayercolor, \translayerlinetype]
		(\translayerrightstartx,\translayerrightstarty)
		-- ++(0,-\translayerrightdeltay);}{}
	
	%%% Random half space
	\ifthenelse{\halfspacetype=2}{
		
		\path[fill=\halfspacefillcolor]
		decorate[decoration={random steps, segment length=0.25cm}]
		{(-\supportwidth,0) ++(-\hsdistleft,0) ++(-\leftsoildist,\leftwallh)
			-- ++(0,-\supportheight-\hsdistbottom)
			-- ++(0,-\leftsoildepth)..controls ++(0,-\leftcontroly) and
			 ++(-\leftcontrolx,0) .. (-\supportwidth-\hsdistleft,-\soilbelowfoundtof)
			-- ++(\supportwidth,0) -- ++(\supportwidth,0)
			-- ++(\hsdistleft,0) -- ++(\hsdistright,0)
			-- ++(\buildingwidth,0)	..controls ++(\rightcontrolx,0)
			and ++(0,-\rightcontroly) .. (\rightsoilx,\rightsoily)
			-- ++(0,\hsdistbottom+\supportheight) -- ++(0,\rightsoildepth)}
		-- ++(-\rightsoildist,0) -- ++(0,-\rightwallh)
		-- ++(0,-\hsdistbottom)-- ++(0,-\supportheight)
					-- ++(-\buildingwidth,0)
		-- ++(-\hsdistleft,0) -- ++(-\hsdistright,0)
		-- ++(-\supportwidth,0) -- ++(-\supportwidth,0)
		-- ++(0,\leftwallh)-- ++(0,\hsdistbottom) -- ++(0,\supportheight)		
		-- cycle;
				
		\draw[line width=\hsinnerlinethk]
		(-\supportwidth,\thinlinethk/2) ++(-\hsdistleft,\leftwallh)
		--++(0,-\leftwallh)	--++(0,-\hsdistbottom) --++(0,-\supportheight)
		--++(0,-\thinlinethk/2) --++(\buildingwidth,0)
		--++(\hsdistleft,0)  --++(\hsdistright,0)
		-- ++(\supportwidth,0) -- ++(\supportwidth,0)
		--++(0,\thinlinethk/2) --++(0,\hsdistbottom) --++(0,\supportheight)
		--++(0,\rightwallh);
		%\draw(-5cm,-\soilbelowfoundtof) --++(15cm,0);
		}{}
	
	%% White rectangle
	\ifthenelse{\NOT \subfloors=0}{
		\path[fill=white](0,\basewalldepth+1pt) -- ++(\buildingwidth,0)
		--++(0,-\basewalldepth-1pt) -- ++(-\buildingwidth,0) -- cycle;}{}
	
	%%% Piles
	\ifthenelse{\showpiles=1}{
		\foreach \pileind in {1,...,{\numberofpiles}}
		{\pilesupport[x coordinate=\pilecoordx{\pileind},
			y coordinate=\pilecoordy,
			pile depth=\piledepth, pile diameter=\pilediameter,
			pile line thickness=\pilelinethickness, fill color=\pilefillcolor,
			draw boundary line=\showpbline,
			boundary line thickness=\pblinet,
			boundary line color=\pblinecolor,
			boundary line type=\pblinetype,
			horizontal show springs= \showphorizontalsprings,
horizontal number of springs = \phorspringnumber,
horizontal spring direction = \phorspringdir,
horizontal space between springs = \phorspbtwspr,
horizontal space = \phorspringspace,
horizontal start ratio = \phorspringstartratio,
horizontal end ratio = \phorspringendratio,
horizontal spring text = \phorspringtext,
horizontal text color = \phortextcolor,
horizontal text shiftx= \phortextshiftx,
horizontal text shifty= \phortextshifty,
horizontal spring prelength ratio= \phorprelenratio,
horizontal spring postlength ratio= \phorpostlenratio,
horizontal spring segment= \phorsegm,
horizontal spring width= \phorsprwid,
horizontal spring scale= \phorsprscale,
horizontal spring line thickness= \phorsprlinethk,
horizontal spring color = \phorsprcolor,
horizontal support width= \phorsuppwidth,
horizontal support depth= \phorsuppdepth,
horizontal support line thickness= \phorsupplinethk,
horizontal show support shade= \phorshowsuppshade,
horizontal support shade color= \phorsuppshadecol,
vertical show springs= \showpverticalsprings,
vertical number of springs = \pverspringnumber,
vertical space = \pverspringspace,
vertical start ratio = \pverspringstartratio,
vertical end ratio = \pverspringendratio,
vertical spring text = \pverspringtext,
vertical text color=\pvertextcolor
vertical text shiftx= \pvertextshiftx,
vertical text shifty= \pvertextshifty,
vertical spring length= \pverspringlength,
vertical spring prelength ratio= \pverprelenratio,
vertical spring postlength ratio= \pverpostlenratio,
vertical spring width= \pverampl,
vertical spring segment= \pversegm,
vertical spring scale= \pverspringscale,
vertical spring line thickness= \pverspringthk,
vertical spring color= \pverspringcolor,
vertical support width= \pversuppwidth,
vertical support depth = \pversuppdepth,
vertical support line thickness = \pversupplinethk,
vertical show support shade= \pvershowsuppshade,
vertical support shade color= \pversuppshadecol,
axial show spring= \showpaxialspring,
axial spring text = \paxialspringtext,
axial text color=\paxialtextcolor,
axial text shiftx= \paxialtextshiftx,
axial text shifty= \paxialtextshifty,
axial spring length=\paxialspringlength,
axial spring prelenratio=\paxialspringprelenratio,
axial spring postlength ratio=\paxialspringpostlenratio,
axial spring segment=\paxialspringsegm,
axial spring width=\paxialspringwidth,
axial spring line thickness=\paxialspringlinethk,
axial spring color=\paxialspringcolor,
axial support width= \paxialsuppwidth,
axial support depth = \paxialsuppdepth,
axial support line thickness = \paxialsupplinethk,
axial show support shade= \paxialshowsuppshade,
axial support shade color= \paxialsuppshadecol,]}
	}{}
	
	
	%%% Left Wall Springs
	\ifthenelse{\leftwshowsprings=1}{
	\framedistributedspring[startx = \basewallstartx,
	starty = -\supportheight,
	endx = \basewallstartx,
	endy = \leftwallh,
	number of springs =\leftwspringnumber,
	space =\leftwspringspace,
	start ratio =\leftwspringstartratio,
	end ratio =\leftwspringendratio,
	spring text =\leftwspringtext,
	text shiftx=\leftwtextshiftx,
	text shifty=\leftwtextshifty,
	text location=\leftwspringtextloc,
	spring length=\leftwspringlength,
	spring prelength ratio=\leftwprelenratio,
	spring postlength ratio=\leftwpostlenratio,
	spring width=\leftwampl,
	spring segment=\leftwsegm,
	spring scale=\leftwspringscale,
	spring line thickness=\leftwspringthk,
	spring color=\leftwspringcolor,
	support width=\leftwsuppwidth,
	support depth =\leftwsuppdepth,
	support line thickness =\leftwsupplinethk,
	show support shade=\leftwshowsuppshade,
	support shade color=\leftwsuppshadecol,]}
	
	%%% Right Wall Springs
	\ifthenelse{\rightwshowsprings=1}{
	\framedistributedspring[startx = \rightwallspringstartx,
	starty = \rightwallspringstarty,
	endx = \rightwallspringendx,
	endy = \rightwallspringendy,
	number of springs =\rightwspringnumber,
	space =\rightwspringspace,
	start ratio =\rightwspringstartratio,
	end ratio =\rightwspringendratio,
	spring text =\rightwspringtext,
	text shiftx=\rightwtextshiftx,
	text shifty=\rightwtextshifty,
	text location=\rightwspringtextloc,
	spring length=\rightwspringlength,
	spring prelength ratio=\rightwprelenratio,
	spring postlength ratio=\rightwpostlenratio,
	spring width=\rightwampl,
	spring segment=\rightwsegm,
	spring scale=\rightwspringscale,
	spring line thickness=\rightwspringthk,
	spring color=\rightwspringcolor,
	support width=\rightwsuppwidth,
	support depth =\rightwsuppdepth,
	support line thickness =\rightwsupplinethk,
	show support shade=\rightwshowsuppshade,
	support shade color=\rightwsuppshadecol,]}

	%%% Foundations Springs
	\ifthenelse{\foundshowsprings=1}{
	\framedistributedspring[startx = \rightwallspringendx,
	starty = \rightwallspringendy,
	endx = \basewallstartx,
	endy = -\supportheight,
	number of springs =\foundspringnumber,
	space =\foundspringspace,
	start ratio =\foundspringstartratio,
	end ratio =\foundspringendratio,
	spring text =\foundspringtext,
	text shiftx=\foundtextshiftx,
	text shifty=\foundtextshifty,
	text location=\foundspringtextloc,
	spring length=\foundspringlength,
	spring prelength ratio=\foundprelenratio,
	spring postlength ratio=\foundpostlenratio,
	spring width=\foundampl,
	spring segment=\foundsegm,
	spring scale=\foundspringscale,
	spring line thickness=\foundspringthk,
	spring color=\foundspringcolor,
	support width=\foundsuppwidth,
	support depth =\foundsuppdepth,
	support line thickness =\foundsupplinethk,
	show support shade=\foundshowsuppshade,
	support shade color=\foundsuppshadecol,]}
	

\ifthenelse{\showleftpinsup=1}{
\ifthenelse{\leftstory>0}{
\foreach \ij in {1,...,{\leftstory}}{	
	\pinsupport[%
	triangle width=\lefttriw,
	show support shade=\showleftpinsuppshade,
	support shade color=\leftpinsuppshadecolor,
	line width=\leftlinew,
	line depth=\leftlined,
	line thickness=\leftlinet,
	radius ratio=\leftradratio,
	x value= -\supportwidth,
	y value=\y{\ij},
	show circle=\leftshowcircle,
	rotation=\leftrotpin]
}
\pinsupport[%
triangle width=\lefttriw,
show support shade=\showleftpinsuppshade,
support shade color=\leftpinsuppshadecolor,
line width=\leftlinew,
line depth=\leftlined,
line thickness=\leftlinet,
radius ratio=\leftradratio,
x value=-\supportwidth,
y value=\y{\leftlevel}-\groundbeamshifty,
show circle=\leftshowcircle,
rotation=\leftrotpin]}}


\ifthenelse{\showrightpinsup=1}{
\ifthenelse{\rightstory>0}{
\foreach \ij in {1,...,{\rightstory}}{
	\pinsupport[%
	triangle width=\righttriw,
	show support shade=\showrightpinsuppshade,
	support shade color=\rightpinsuppshadecolor,
	line width=\rightlinew,
	line depth=\rightlined,
	line thickness=\rightlinet,
	radius ratio=\rightradratio,
	x value=\rightwallspringstartx,
	y value=\y{\ij},
	show circle=\rightshowcircle,
	rotation=\rightrotpin]
}
\pinsupport[%
triangle width=\righttriw,
show support shade=\showrightpinsuppshade,
support shade color=\rightpinsuppshadecolor,
line width=\rightlinew,
line depth=\rightlined,
line thickness=\rightlinet,
radius ratio=\rightradratio,
x value=\rightwallspringstartx,
y value=\y{\rightlevel}-\groundbeamshifty,
show circle=\rightshowcircle,
rotation=\rightrotpin]}}


\ifthenelse{\showbottompinsup=1}{
\foreach \ijkk in {1,...,\nbotpin}{
	\pinsupport[%
	triangle width=\bottomtriw,
	show support shade=\showbottompinsuppshade,
	support shade color=\bottompinsuppshadecolor,
	line width=\bottomlinew,
	line depth=\bottomlined,
	line thickness=\bottomlinet,
	radius ratio=\bottomradratio,
	x value=\botpinx{\ijkk},
	y value=-\supportheight,
	show circle=\bottomshowcircle,
	rotation=\bottomrotpin];}
}{};

	%%% Retaining walls and mat foundations
	\ifthenelse{\NOT \subfloors=0}{
	\filldraw[line width=\baselinet, fill=\supportfillcolor, draw=\supportlinecolor]
	(\basewallstartx,\leftwallh) -- ++(\supportwidth,0)
	-- ++(0,-\leftwallh) -- ++(\buildingwidth,0)
	-- ++(0,\rightwallh) -- ++(\supportwidth,0)
	-- ++(0,-\rightwallh)-- ++(0,-\supportheight)
	-- (\basewallstartx,-\supportheight)
	-- cycle;}{}
	
	%%%Soil-Structure Interface
	\ifthenelse{\showssinter=1}{
		\draw[line width=\ssinterlinet, \ssintercolor, \ssinterlinetype]
		(\ssinterleftstartx,\ssinterleftstarty) -- ++(0,-\ssinterleftdeltay) 
		-- ++ (\ssinterdeltax,0) -- ++(0,\ssinterrightdeltay);}{}
	
	%%% Top thin line
	\ifthenelse{\NOT \halfspacetype=0}{
	\draw[line width=\thinlinethk, color=black]
	(0,\leftwallh) ++(-\supportwidth,0) ++(-\hsdistleft,0)
	-- ++(-\leftsoildist-\translayerlinet,0);
	
	\draw[line width=0.5pt, color=black]
	(\buildingwidth,\rightwallh) ++(\supportwidth,0) ++(\hsdistleft,0)
	-- ++(\rightsoildist+\translayerlinet,0);
	}
	
	%% Engineering Bedrock
	\ifthenelse{\showengbedrock=1}{
		\fill[\engbedrockfillcolor]
		(\engbedrockstartx, \engbedrockstarty) rectangle
		++(\engbedrockdeltax, \engbedrockdeltay);
		
		\draw[line width=\engbedrocklinewidth]
		(\engbedrockstartx,\engbedrockstarty) ++(0, \engbedrockdeltay)
		-- ++(\engbedrockdeltax,0);
		
		\node[align=center, scale=1] at
		(\engbedrockmidx, \engbedrockmidy) {\engbedrocktext};}{}
	
}{}




%%Substructure - Same for both deflected shape and undeflected shape
\ifthenelse{\subfloors>0}{
	
	%Draw the beam at the ground level
	\draw [line width = \groundbeamlinet, \beamcolor]
	(\x{1}+\baselinet/2, \y{\deflstart}-\groundbeamshifty)
	-- (\x{\columnnumber}-\baselinet/2, \y{\deflstart}-\groundbeamshifty);
	
	%Draw the beams except the ground level beam
	\ifthenelse{\subfloors>1}{
	\foreach \iii in {2,...,{\subfloors}}
	{\draw [blue, line width = \beamlinet, \beamcolor]
	(\x{1}+\baselinet/2,\y{\iii})
	-- (\x{\columnnumber}-\baselinet/2,\y{\iii});}
	}{}
	
	%% Draw first column - No Need to Draw when bays are reduced
	\ifthenelse{\leftbays=0}{
	\ifthenelse{\superstorynumber>0 \OR \leftopenstory>0}{
	\draw [line width = \collinet, \columncolor]
	(\x{1},\y{1}+\baselinet/2)
	-- (\x{1}, \y{\deflstart}+\baselinet/2);
	
	\draw[line width=\baselinet, color=\supportlinecolor]
	(\x{1}, \y{1}) -- ++(-\collinet/2,0)
	-- ++(-\baselinet/2,0) -- ++(0,\leftwallh);}{}}{}

	%% When left open - same as above
	\ifthenelse{\leftbays>0}{
	\ifthenelse{\leftopenstory>0}{
	
	\draw [line width = \collinet, \columncolor]
	(\x{1},\y{1}+\baselinet/2)
	-- (\x{1}, \y{\deflstart}+\baselinet/2);
	
	\draw[line width=\baselinet, color=\supportlinecolor]
	(\x{1}, \y{1}) -- ++(-\collinet/2,0)
	-- ++(-\baselinet/2,0) -- ++(0,\leftwallh);}{}}{}

	
	%% Draw the columns between the first and the last column
	\foreach \j in {2,...,{\ncolmo}}
	{\draw [line width = \collinet, \columncolor]
	(\x{\j}, \y{1}+\baselinet/2)
	-- (\x{\j}, \y{\deflstart}+\baselinet/2);}
	
	%% Draw the last column - No Need to Draw when bays are reduced
	\ifthenelse{\rightbays=0}{
	\ifthenelse{\superstorynumber>0 \OR \rightopenstory>0}{
	
	\draw [line width = \collinet, \columncolor]
	(\x{\columnnumber}, \y{1}+\baselinet/2)
	-- (\x{\columnnumber}, \y{\deflstart}+\baselinet/2);
	
	\draw[line width=\baselinet, color=\supportlinecolor]
	(\x{\columnnumber}, \y{1}) -- ++(+\collinet/2,0)
	-- ++(\baselinet/2,0)  -- ++(0,\rightwallh);}{}}{}	

	%% When Right Open
	\ifthenelse{\rightbays>0}{
	\ifthenelse{\rightopenstory>0}{
	
	\draw [line width = \collinet, \columncolor]
	(\x{\columnnumber}, \y{1}+\baselinet/2)
	-- (\x{\columnnumber}, \y{\deflstart}+\baselinet/2);
	
	\draw[line width=\baselinet, color=\supportlinecolor]
	(\x{\columnnumber}, \y{1}) -- ++(+\collinet/2,0)
	-- ++(\baselinet/2,0)  -- ++(0,\rightwallh);}{}}{}	
}{} %% Closure for substructure - same as deflected and undeflected shape


%\draw (0,0)-- (0.5cm,2cm);
%%% Draw Frame - Undeflected shape
\ifthenelse{\showdefl=0}{
%%% Superstructure
\ifthenelse{\superstorynumber>0}{
%% Draw beams up the top level minus one
\ifthenelse{\superstorynumber>1}{
\foreach \iii in {\deflstartplusone,...,{\nlevmo}}
{\draw [\beamcolor, line width = \beamlinet]
(\x{\startcol},\y{\iii}) -- (\x{\endcol},\y{\iii});}
}{}

%% Draw the beam the the top floor
\draw [line width = \beamlinet, \beamcolor]
(\x{\startcol}-\collinet/2, \y{\levelnumber})
-- (\x{\endcol}+\collinet/2, \y{\levelnumber});

%% Draw the first column
\draw [line width = \collinet, \columncolor]
(\x{\startcol},\y{\deflstart}-\baselinet/2)
-- (\x{\startcol}, \y{\levelnumber}+\beamlinet/2);

%% Columns other than first and last columns
\foreach \j in {\startcol,...,{\endcol}}
{\draw [line width = \collinet, \columncolor]
(\x{\j},\y{\deflstart}-\baselinet/2)
-- (\x{\j},\y{\levelnumber}+\beamlinet/2);}
}{}

%%%% Shear Wall
\ifthenelse{\showshearwall=1}{
\filldraw[fill=\shearwallfillcolor, line width=\shearwalllinet]
(\shearwallstartx, \shearwallstarty)
rectangle ++(\shearwalldeltax,\shearwalldeltay);
}{}

}{} %%% Closure Undeflected Shape





%\node at (10,10) {\deflstart};
%\node at (50,50) {\deflstartplusone};
%\node at (100,100) {\superstorynumber};
%%% Draw Frame - Deflected shape
\ifthenelse{\showdefl=1}{
	
%%%Superstructure

%% If the superstory number is different then zero, first draw the first story
\ifthenelse{\superstorynumber>0}{
%%Beams at the second level only
\foreach \ibeam in {\startbeam,...,{\endbeam}}
{
	\draw [red, line width = \beamlinet, \beamcolor]
	(\x{\ibeam}+\deflect{\deflstartplusone}, \y{\deflstartplusone})
	.. controls +(\defparbeamy,-\defparbeamx) and +(-\defparbeamy, \defparbeamx) ..
	(\usevar\x{\ibeam+1} + \deflect{\deflstartplusone}, \y{\deflstartplusone});
}
%% Columns at the first story only
\foreach \icol in {\startcol,...,{\endcol}}
{
	\draw [line width = \collinet, \columncolor, opacity=1]
	(\x{\icol}+\deflect{\deflstart},\y{\deflstart})
	.. controls +(0.0cm, \defbase) and +(-\defparx, -\defpary) ..
	(\x{\icol}+\deflect{\deflstartplusone}+\fixcolx, \y{\deflstartplusone}+\fixcoly);
}}{}


%% If story number is larger than two, draw the storys other than first and last story
\ifthenelse{\superstorynumber>1}{

\foreach \istory in {\deflstartplusone,...,{\storynumber}}
{
	\foreach \ibeam in {\startbeam,...,{\endbeam}}
	{
		\draw [blue, line width = \beamlinet, \beamcolor]
		(\x{\ibeam}+\usevar\deflect{\istory+1}, \usevar\y{\istory+1})
		.. controls +(\defparbeamy,-\defparbeamx) and +(-\defparbeamy, +\defparbeamx)
		.. (\usevar\x{\ibeam+1}+\usevar\deflect{\istory+1},
		\usevar\y{\istory+1});
	}
	\foreach \icol in {\startcol,...,{\endcol}}
	{
		\draw [line width = \collinet, \columncolor]
		(\x{\icol}+\deflect{\istory}+\fixcolx, \y{\istory}+\fixcoly)
		.. controls +(\defparx, \defpary) and +(-\defparx, -\defpary) ..
		(\usevar\x{\icol}+\usevar\deflect{\istory+1}+\fixcolx, \usevar\y{\istory+1}+\fixcoly);
	}
}


}{}

}{} %% Closure for deflected shape



%%% Mark for Superstructure
\ifthenelse{\superstorynumber>0}{
	\ifthenelse{\showmarksuper=1}{
		\draw[rounded corners=\marksuperrad,
		line width=\marksuperlinet, color=\marksuperlinecolor, \marksuperlinetype]
		(\marksuperleftstartx,\marksuperleftstarty)
		rectangle ++(\marksuperdeltax,\marksuperdeltay);}{}}{}


%%%% Different Support Types
%% 0. No Supports
\ifthenelse{\showsupports=0}{
}{}

%% 1. Individual fixed supports
\ifthenelse{\showsupports=1}{
	\foreach \j in {1,...,{\columnnumber}}
	{
		\fill [\supportfillcolor] (\x{\j},\y{1}) +({-\supportwidth/2},-\supportheight)
		rectangle +({\supportwidth/2},\y{1});
		\draw [line width = \baselinet, color=\supportlinecolor]
		(\x{\j},\y{1}) +({-\supportwidth/2},\y{1}) -- +({\supportwidth/2},\y{1});
	}
	\ifthenelse{\showshearwall=1}{
		\fill[\supportfillcolor]
		(\shearwallsupportstartx, \y{1})
		rectangle ++(\shearwallsupportdeltax,-\supportheight);
		
		\draw[line width=\baselinet, color=\supportlinecolor]
		(\shearwallsupportstartx, \y{1})
		-- ++(\shearwallsupportdeltax,0);
	}{}
}{}

%% 2. Support as continuous plate
\ifthenelse{\showsupports=2}{
	\fill [fill=\supportfillcolor] (\rigbasestartx, 0) rectangle (\rigbaseendx,-\supportheight);
	\draw [line width = \baselinet, color=\supportlinecolor] (\rigbasestartx, 0) -- (\rigbaseendx,0);
}{}


%% 3. Support as ...
\ifthenelse{\showsupports=3}{
	\foreach \j in {1,...,{\columnnumber}}
	{\fill [fill=\supportfillcolor, draw=\supportlinecolor, line width = \baselinet]
		(\rigbasestartx, 0) rectangle (\rigbaseendx,-\supportheight);}
}{}


%%% 4. Base Isolation
\ifthenelse{\showsupports=4}{
	
	\foreach \j in {1,...,{\numberofisolators}}
	{\filldraw [fill=\isofillcolor, draw=\isolinecolor, line width=\isolinet]
		(\xiso{\j},\isoboty) ++(-\isolationwidth/2-\isodeflx,0) -- ++(\isolationwidth,0)
		-- ++(\isodeflx,\isolationdepth) -- ++(-\isolationwidth,0) -- cycle;}
		
		
	\fill [fill=\isofloorfillcolor, draw=\isofloorlinecolor, line width = \isofloorlinethk]
	(\rigbasestartx, 0) rectangle 	(\rigbaseendx,-\supportheight);
	
	\fill [fill=\supportfillcolor]
	(\foundstartx, \foundboty) rectangle (\foundendx,\foundtopy);
	\draw [line width = \baselinet, color=\supportlinecolor]
	(\foundstartx, \foundtopy) -- (\foundendx, \foundtopy);
	
}{}

% Fix for shear wall foundations
\ifthenelse{\showshearwall=1}{
	\ifthenelse{\NOT \showsupports=1}{
		\ifthenelse{\NOT \showsupports=4}{
		\draw[line width=\shearwalllinet, \supportfillcolor]
		(\shearwallstartx-\shearwalllinet/2, \shearwallstarty)
		--++(\shearwalldeltax+\shearwalllinet,0);
		\draw[line width=\baselinet, color=\supportlinecolor]
		(\shearwallstartx-\shearwalllinet/2,\y{1})
		--++(\shearwalldeltax+\shearwalllinet,0);}{}}{}}{}



%\draw (0,0) -- node{\showsupports} (5cm,5cm) ;

%% Fix for mat foundation
\ifthenelse{\showsupports=5}{
\ifthenelse{\subfloors=0}{
	\filldraw[line width=\baselinet, fill=\supportfillcolor, draw=\supportlinecolor]
	(\basewallstartx,\leftwallh) -- ++(\supportwidth,0)
	-- ++(0,-\leftwallh) -- ++(\buildingwidth,0)
	-- ++(0,\rightwallh) -- ++(\supportwidth,0)
	-- ++(0,-\rightwallh)-- ++(0,-\supportheight)
	-- (\basewallstartx,-\supportheight)
	-- cycle;}}{}





%%%% Masses

\ifthenelse{\showmass=1}{
%% Masses for superstructure - Undeflected Shape
\ifthenelse{\showdefl=0}{
\ifthenelse{\superstorynumber>0}{
\foreach \iii in {\useeval{2+\subfloors},...,{\levelnumber}}{
\foreach \j in {\startcol,...,{\endcol}}{

\ifthenelse{\showshearwall=1}{
	
	\ifthenelse{\j=\shearwallstartcolumn}{
	\shade[ball color=\masscolorsuper]
	(\x{\j}-\collinet/2,\y{\iii}) circle (\massrad);}
	
	\ifthenelse{\j=\shearwallendcolumn}{
	\shade[ball color=\masscolorsuper]
	(\x{\j}+\collinet/2,\y{\iii}) circle (\massrad);}
	
	\ifthenelse{\NOT{\j=\shearwallstartcolumn}
		\AND \NOT{\j=\shearwallendcolumn}}{
	\shade[ball color=\masscolorsuper]
	(\x{\j},\y{\iii}) circle (\massrad);}
	
}{
	\shade[ball color=\masscolorsuper] (\x{\j}, \y{\iii}) circle (\massrad);
}
}}}{}
}{}


%% Masses for superstructure - Deflected Shape
\ifthenelse{\showdefl=1}{
\ifthenelse{\superstorynumber>0}{
	\foreach \iii in {\useeval{2+\subfloors},...,{\levelnumber}}{
	\foreach \j in {\startcol,...,{\endcol}}{
	\shade[ball color=\masscolorsuper] (\x{\j} + \deflect{\iii}, \y{\iii}) circle (\massrad);
}}}}{}


%%% Masses for substructure (Supports=5)
\ifthenelse{\showsupports=5}{
\ifthenelse{\subfloors>0}{
\foreach \iii in {2,...,\useeval{\deflstart}}{
\foreach \j in {2,...,\useeval{\columnnumber-1}}{
%% Masses at ground level
\ifthenelse{\iii=\deflstart}{
	
	\ifthenelse{\showshearwall=1}{
		
		\ifthenelse{\j=\shearwallstartcolumn}{
			\shade[ball color=\masscolorsub]
			(\x{\j}-\collinet/2,\y{\iii}-\groundbeamshifty) circle (\massrad);}
		
		\ifthenelse{\j=\shearwallendcolumn}{
			\shade[ball color=\masscolorsub]
			(\x{\j}+\collinet/2,\y{\iii}-\groundbeamshifty) circle (\massrad);}
		
		\ifthenelse{\NOT{\j=\shearwallstartcolumn}
			\AND \NOT{\j=\shearwallendcolumn}}
		{\shade[ball color=\masscolorsub]
			(\x{\j},\y{\iii}-\groundbeamshifty) circle (\massrad);}
		
	}{
		\shade[ball color=\masscolorsub] (\x{\j}, \y{\iii}-\groundbeamshifty) circle (\massrad);
	}
}{
	
	\ifthenelse{\showshearwall=1}{
		
		\ifthenelse{\j=\shearwallstartcolumn}{
			\shade[ball color=\masscolorsub]
			(\x{\j}-\collinet/2,\y{\iii}) circle (\massrad);}
		
		\ifthenelse{\j=\shearwallendcolumn}{
			\shade[ball color=\masscolorsub]
			(\x{\j}+\collinet/2,\y{\iii}) circle (\massrad);}
		
		\ifthenelse{\NOT{\j=\shearwallstartcolumn}
			\AND \NOT{\j=\shearwallendcolumn}}
		{\shade[ball color=\masscolorsub]
			(\x{\j},\y{\iii}) circle (\massrad);}
	}{
		\shade[ball color=\masscolorsub] (\x{\j}, \y{\iii}) circle (\massrad);
	}
	
}

}}}}{}

}{} %%% Closure show mass


%%% White Shading for Superstructure
\ifthenelse{\showsupports=5}{
	\ifthenelse{\showsupershade=1}{
		\fill[fill=white, opacity=\supershadeopacity]
		(\supershadestartx,\supershadestarty)
		rectangle ++(\supershadedeltax,\supershadedeltay);}{}


%%% Mark for Substructure
\ifthenelse{\showmarkss=1}{
\draw[rounded corners=\markssrad,
line width=\marksslinet, color=\marksslinecolor, \marksslinetype]
(\markssleftstartx,\markssleftstarty)
rectangle ++(\markssdeltax,-\markssdeltay);}{}
}{}

%%% Global Axes
\ifthenelse{\showaxes=1}{
\tikzset{>={Classical TikZ Rightarrow[scale=1, width=3, length=5, bend]}};
\draw [{->}] (\Xaxesstartx,\Xaxesstarty) -- +(\axeslenX,0) node[above]{$X$};
\draw [{->}] (\Yaxesstartx,\Yaxesstarty) -- +(0,\axeslenY) node[right]{$Y$};
}{}



%%% Degrees of Freedom

%% DOF at only one joint
\ifthenelse{\showdof=1}{
\dofs[startx = \dofxx, starty = \dofyy,
  length x = \arrlen,  length y = \arrlen,
  offset x=\dofoffsetx, offset y=\dofoffsety,
  offset rotation xdir=\dofrotoffsetx, offset rotation ydir=\dofrotoffsety,
  start angle = \rotdofstartangle, end angle = \rotdofendangle, radius = \arrrad,
  xstring = \dofxstr, ystring = \dofystr, rstring = \dofrstr,
  font size = \normalsize, font scale = \doftextratio,
  rotation=0, font rotation=0,
  font rotation for x = \dofxrotation,
  font rotation for y = \dofyrotation,
  font rotation for r = \dofrrotation,
  dof arrow size ratio=\dofarrowratio,
  show dofx = \shodofx, show dofy = \shodofy, show dofr = \shodofr,
  dofx position=\dofposx, dofy position=\dofposy, dofr position=\dofposr,
  dofx inner sep=\dofinnersepx, dofy inner sep=\dofinnersepx,
  dofr inner sep=\dofinnersepr,]  
}{}

\ifthenelse{\showdof=2}{

\ifthenelse{\showsupports=5}{
\foreach \iii in {\useeval{2+\subfloors},...,{\levelnumber}}
{\foreach \j in {1,...,{\columnnumber}}{

\ifthenelse{\shodofx=1}{
	\ifthenelse{\shodofy=1}{
		\ifthenelse{\shodofr=1}{
			\tikzmath{
				\dofcounterx=(\iii-2)*\columnnumber*3+ (\j-1)*3 + 1;
				\dofcountery=\dofcounterx+1);
				\dofcounterz=\dofcountery+1;}
		}{}	
	}{}	
}{}

\ifthenelse{\shodofx=1}{
	\ifthenelse{\shodofy=1}{
		\ifthenelse{\shodofr=0}{
			\tikzmath{
				\dofcounterx=(\iii-2)*\columnnumber*2+ (\j-1)*2 + 1;
				\dofcountery=\dofcounterx+1);}
		}{}	
	}{}	
}{}

\ifthenelse{\shodofx=1}{
	\ifthenelse{\shodofy=0}{
		\ifthenelse{\shodofr=1}{
			\tikzmath{
				\dofcounterx=(\iii-2)*\columnnumber*2+ (\j-1)*2 + 1;
				\dofcounterz=\dofcounterx+1);}
		}{}	
	}{}	
}{}

\ifthenelse{\shodofx=0}{
	\ifthenelse{\shodofy=1}{
		\ifthenelse{\shodofr=1}{
			\tikzmath{
				\dofcountery=(\iii-2)*\columnnumber*2+ (\j-1)*2 + 1;
				\dofcounterz=\dofcountery+1);}
		}{}	
	}{}	
}{}

\ifthenelse{\shodofx=0}{
	\ifthenelse{\shodofy=0}{
		\ifthenelse{\shodofr=1}{
			\tikzmath{
				\dofcounterz=(\iii-2)*\columnnumber*1+ (\j-1)*1 + 1;}
		}{}	
	}{}	
}{}


\dofs[startx = \x{\j}+\deflect{\iii}, starty = \y{\iii},
  length x = \arrlen,  length y = \arrlen,
  offset x=\dofoffsetx, offset y=\dofoffsety,
  offset rotation xdir=\dofrotoffsetx, offset rotation ydir=\dofrotoffsety,
  start angle = \rotdofstartangle, end angle = \rotdofendangle, radius = \arrrad,
  xstring = \dofcounterx, ystring = \dofcountery, rstring = \dofcounterz,
  font size = \normalsize, font scale = \doftextratio,
  rotation=0, font rotation=0,
  font rotation for x = \dofxrotation,
  font rotation for y = \dofyrotation,
  font rotation for r = \dofrrotation,
  dof arrow size ratio=\dofarrowratio,
  show dofx = \shodofx, show dofy = \shodofy, show dofr = \shodofr,
  dofx position=\dofposx, dofy position=\dofposy, dofr position=\dofposr,
  dofx inner sep=\dofinnersepx, dofy inner sep=\dofinnersepx,
  dofr inner sep=\dofinnersepr,]  

}}

\foreach \iii in {2,...,\useeval{\subfloors+1}}
{\foreach \j in {2,...,\useeval{\columnnumber-1}}{
		
\ifthenelse{\shodofx=1}{
	\ifthenelse{\shodofy=1}{
		\ifthenelse{\shodofr=1}{
			\tikzmath{
				\dofcounterx=(\iii-2)*\columnnumber*3+ (\j-1)*3 + 1;
				\dofcountery=\dofcounterx+1);
				\dofcounterz=\dofcountery+1;}
		}{}	
	}{}	
}{}

\ifthenelse{\shodofx=1}{
	\ifthenelse{\shodofy=1}{
		\ifthenelse{\shodofr=0}{
			\tikzmath{
				\dofcounterx=(\iii-2)*\columnnumber*2+ (\j-1)*2 + 1;
				\dofcountery=\dofcounterx+1);}
		}{}	
	}{}	
}{}

\ifthenelse{\shodofx=1}{
	\ifthenelse{\shodofy=0}{
		\ifthenelse{\shodofr=1}{
			\tikzmath{
				\dofcounterx=(\iii-2)*\columnnumber*2+ (\j-1)*2 + 1;
				\dofcounterz=\dofcounterx+1);}
		}{}	
	}{}	
}{}

\ifthenelse{\shodofx=0}{
	\ifthenelse{\shodofy=1}{
		\ifthenelse{\shodofr=1}{
			\tikzmath{
				\dofcountery=(\iii-2)*\columnnumber*2+ (\j-1)*2 + 1;
				\dofcounterz=\dofcountery+1);}
		}{}	
	}{}	
}{}

\ifthenelse{\shodofx=0}{
	\ifthenelse{\shodofy=0}{
		\ifthenelse{\shodofr=1}{
			\tikzmath{
				\dofcounterz=(\iii-2)*\columnnumber*1+ (\j-1)*1 + 1;}
		}{}	
	}{}	
}{}


\dofs[startx = \x{\j}+\deflect{\iii}, starty = \y{\iii},
  length x = \arrlen,  length y = \arrlen,
  offset x=\dofoffsetx, offset y=\dofoffsety,
  offset rotation xdir=\dofrotoffsetx, offset rotation ydir=\dofrotoffsety,
  start angle = \rotdofstartangle, end angle = \rotdofendangle, radius = \arrrad,
  xstring = \dofcounterx, ystring = \dofcountery, rstring = \dofcounterz,
  font size = \normalsize, font scale = \doftextratio,
  rotation=0, font rotation=0,
  font rotation for x = \dofxrotation,
  font rotation for y = \dofyrotation,
  font rotation for r = \dofrrotation,
  dof arrow size ratio=\dofarrowratio,
  show dofx = \shodofx, show dofy = \shodofy, show dofr = \shodofr,
  dofx position=\dofposx, dofy position=\dofposy, dofr position=\dofposr,
  dofx inner sep=\dofinnersepx, dofy inner sep=\dofinnersepx,
  dofr inner sep=\dofinnersepr,] 	
	
}}
}


{
\foreach \iii in {2,...,{\levelnumber}}
{\foreach \j in {1,...,{\columnnumber}}{


\ifthenelse{\shodofx=1}{
	\ifthenelse{\shodofy=1}{
		\ifthenelse{\shodofr=1}{
			\tikzmath{
				\dofcounterx=(\iii-2)*\columnnumber*3+ (\j-1)*3 + 1;
				\dofcountery=\dofcounterx+1);
				\dofcounterz=\dofcountery+1;}
		}{}	
	}{}	
}{}

\ifthenelse{\shodofx=1}{
	\ifthenelse{\shodofy=1}{
		\ifthenelse{\shodofr=0}{
			\tikzmath{
				\dofcounterx=(\iii-2)*\columnnumber*2+ (\j-1)*2 + 1;
				\dofcountery=\dofcounterx+1);}
		}{}	
	}{}	
}{}

\ifthenelse{\shodofx=1}{
	\ifthenelse{\shodofy=0}{
		\ifthenelse{\shodofr=1}{
			\tikzmath{
				\dofcounterx=(\iii-2)*\columnnumber*2+ (\j-1)*2 + 1;
				\dofcounterz=\dofcounterx+1);}
		}{}	
	}{}	
}{}

\ifthenelse{\shodofx=0}{
	\ifthenelse{\shodofy=1}{
		\ifthenelse{\shodofr=1}{
			\tikzmath{
				\dofcountery=(\iii-2)*\columnnumber*2+ (\j-1)*2 + 1;
				\dofcounterz=\dofcountery+1);}
		}{}	
	}{}	
}{}

\ifthenelse{\shodofx=0}{
	\ifthenelse{\shodofy=0}{
		\ifthenelse{\shodofr=1}{
			\tikzmath{
				\dofcounterz=(\iii-2)*\columnnumber*1+ (\j-1)*1 + 1;}
		}{}	
	}{}	
}{}


%\node at (\x{\j}+\deflect{\iii},\y{\iii}) {aa};

\dofs[startx = \x{\j}+\deflect{\iii}, starty = \y{\iii},
length x = \arrlen,  length y = \arrlen,
offset x=\dofoffsetx, offset y=\dofoffsety,
offset rotation xdir=\dofrotoffsetx, offset rotation ydir=\dofrotoffsety,
start angle = \rotdofstartangle, end angle = \rotdofendangle, radius = \arrrad,
xstring = \dofcounterx, ystring = \dofcountery, rstring = \dofcounterz,
font size = \normalsize, font scale = \doftextratio,
rotation=0, font rotation=0,
font rotation for x = \dofxrotation,
font rotation for y = \dofyrotation,
font rotation for r = \dofrrotation,
dof arrow size ratio=\dofarrowratio,
show dofx = \shodofx, show dofy = \shodofy, show dofr = \shodofr,
dofx position=\dofposx, dofy position=\dofposy, dofr position=\dofposr,
dofx inner sep=\dofinnersepx, dofy inner sep=\dofinnersepx,
dofr inner sep=\dofinnersepr,] 	
	
}}
}
}{}



\end{scope}
}